\documentclass{article}
\usepackage{amsmath}
\usepackage{amssymb}
\usepackage{graphicx}
\usepackage{hyperref}
\usepackage{mathrsfs}
\usepackage{dsfont}
\usepackage{hyperref}
\usepackage[margin=2cm]{geometry}

\newcommand{\R}{\mathbb{R}}
\renewcommand{\P}{\mathbb{P}}
\newcommand{\A}{\mathbb{A}}
\newcommand{\Z}{\mathbb{Z}}

\newcommand{\br}[1]{\left( #1 \right)}

\begin{document}
\title{Curve Book - William Fulton}
\section*{Capítulo 1 - Conjuntos Algebraicos Afines}
\subsection*{1.1 Preliminarios Algebraicos}
\textbf{Definiciones y Teoremas:}
\\
\\
\textit{Definición 1:} Un dominio entero (integral) es un anillo en donde la ley de cancelación se cumple, es decir, no hay divisores de cero. De forma corta, hablamos de "dominio".
\\
\\
\textit{Definición 2:} Un campo (no cuerpo) es un anillo en donde todo elemento menos el cero es una unidad.
\\
\\
\textit{Definición 3:} Cualquier dominio $R$ posee un campo de fracciones $K$, es decir, un campo que contiene $R$ como un subanillo.
\\
\\
\textit{Definición(es) 4:} Para un anillo $R$, $R[X]$ denota el anillo de polinomios en $R$. El \textit{grado} de un polinomio no nulo $\sum a_i X^{i}$ es el entero más grande $d$ que acompaña al \textit{término principal}. Un polinomio es \textit{mónico} si el término principal es $1$.
\\
\\
(*) El anillo de polinomios en $n$ variables sobre $R$ se denota $R[X_1, \dots , X_n]$
\\
\\
\textit{Definición 5:} Denominamos a un polinomio $F$ \textit{homogeneo} o una $\textit{forma}$ de grado $d$, si todos los coeficientes $a_i$ son nulos excepto para los monomios de grado $d$. (Todos los monomios poseen grado $d$).
\\
\\
(*) Todo polinomio $F$ se puede expresar como $F=F_{0}+F_1 + \dots +F_d$, donde $F_i$ es la \textit{forma} de grado $i$. Si $F_d \neq 0$, $\deg{F}=d$.
\\
\\
\textit{Definición 6:} El anillo $R[X_1 , \dots, X_n]$ es canonicamente isomorfo a $R[X_1 , \dots , X_{n-1}][X_n]$
\\
\\
\textit{Definición 7:} Un elemento $a \in R$ se dice irreducible si no es una unidad o el cero, y para cualquier factorización $a=bc$, $b,c \in R$, entonces $b$ o $c$ es una unidad. (Recordar que irreducible implica primo, pero no al revés!).
\\
\\
\textit{Definición 8:} Un dominio es un $\textit{Dominio de Factorización Única}$ (DFU/UDF) si cada elemento en $R$ puede ser factorizado de forma única. Ver directamente que $R[X]$ es un UDF para $R$ UFD (parte del lema de Gauss).
\\
\\
\textit{Definición 9:} Un \textit{ideal} es propio si $I \neq R$. Un ideal es $\textit{maximal}$ si no está contenido en otro ideal propio. Un ideal es \textit{primo} si es un ideal $I$ tal que si $ab \in I$, entonces $a \in I$ o $b \in I$. 
\\
\\
\textit{Definición 10:} Un ideal es \textit{finitamente generado} si es generado por un conjunto finito $S= \{f_1, \dots, f_n \}$, para esto denotamos $I=(f_1 , \dots, f_n)$. Un ideal es principal si es generado por un elemento. Un dominio en donde cada ideal es principal se denomina \textit{dominio ideal principal} (DIP/PID) (pensar en $\Z$, $\Z[X]$, $k[X]$ para cualquier campo $k$, pero no en $R[X_1, \dots, X_n]$ para $n \geq 2$.)
\\
\\
\textit{Definición 11:} Cada $DIP$ es un \textit{Dominio de Factorización Única} (DFU/UFD). Un ideal $I=(a)$ principal en un UFD es primo si y solo si $a$ es irreducible (o no nulo).
\\
\\
\textit{Definición 12:} Sea $R$ dominio. La \textit{característica} de $R$, $\operatorname{char}{R}$, es el entero más pequeño $p$ tal que $1+\dots+1 \text{ $p$ veces } = 0$
\\
\\
\textit{Definición 12:} Sea $R$ anillo, $a \in R$, $F \in R[X]$, $a$ es una raíz de $F$, entonces $F=(X-a)G$ para un único $G \in R[X]$. Un campo $k$ es $\textit{algebraicamente cerrado}$ si para cualquier polinomio no constante $F \in k[X]$ posee una raíz en $k$
\\
\\
\\
\\
\\
\\
\\
1.1
\\
(a) Sean $F=\sum_{i=0}^{n}a_i f_i(X_1,\dots,X_n)$ y $G=\sum_{j=0}^{m} b_j g_j(X_1, \dots, X_n)$ polinomios homogeneos donde $\deg{f_i}=r$ y $\deg{g_i}=s$ para todo $i,j$. Luego:
\begin{equation*}
    F \cdot G = \sum_{k=0}^{n+m} \left( \sum_{i+j=k}a_i b_i\right) (f_i g_j)(X_1, \dots, X_n)
\end{equation*}
Como $R$ es dominio entero, $\deg{f_i g_i}= \deg{f_i} + \deg{g_j}=r+s$
\\
\\
b) Suponer que $F=F_1 \cdot F_2$ (donde $F_i$ ($i=1,2$) no es trivial al ser $R$ dominio), sean $d_1$y $e_1$ el mayor y menor grado respectivamente de los términos de $F_1$ y sean $d_2$ y $e_2$ respectivamente el mayor y menor grado de $F_2$. El producto de los términos de grado $d_1$ de $F_1$ y $d_2$ de $F_2$ es no nulo y es de grado $d_1 + d_2$. Respectivamente el producto de los términos de grado $e_1$ de $F_1$ y $e_2$ de $F_2$ posee grado $e_1 +e_2$. Como $F$ es homogeneo, $d_1 + d_2 = e_1 + e_2$. Sigue que $0=(d_1 - e_1)+ (d_2 + e_2)$, los sumandos son no-negativos, por lo que $d_1=e_1$ y $d_2 = e_2$. Sigue que $F_1$ y $F_2$ son homogeneos.
\\
\\
1.2
\\
\\
1.3
\\
Como $R$ es un PID, todo ideal está generado por un elemento, entonces $P=(p)$, como es primo, $ab \in (P)$, entonces $a \in (P)$ o $b \in (p)$, esto es equivalente a que $p \mid ab$, entonces $p \mid a$ o $p \mid b$. Para ver que es irreducible, hay que ver si $p=xy$, entonces $x$ o $y$ es una unidad de $R$, supongamos que $x \in (p)$ (recordar la definición de ideal primo!)
\begin{equation*}
\begin{aligned}
    &\Longrightarrow x=k \cdot p \\
    &\Longrightarrow p = kpy \\
    &\longrightarrow 1=ky
\end{aligned}
\end{equation*}
Entonces $y$ es unidad.
\\
\\
b) Para probar que $I$ es maximal, vamos a aprovechar que $R$ es un pid, suponer que hay $I$ tal que $P \subset I \subset R$, entonces $I=(q)$, entonces $(p) \subset (q)$ si y solo si $p=kq$, Pero entonces $k$ o $q$ es unidad, si $k$ es unidad, entonces $q=k^{-1}p$ y 
\\
\\
1.4
\\
Sea $k=1$, entonces $F \in k[X_1]$, entonces $F(a_1)=0$ $\forall a_i \in k$, por lo que $F=0$
\\
\\
Supongamos que se cumple para $k=n-1$ variables
\\
\\
Sea $k=n$. Entonces, $F=\sum F_i(a_1,\dots,a_{n-1})a_{n}^{i}=0$, como $F_i=0$ para $(a_i,\dots,a_{n-1})$, entonces $F=0$ $\forall a_1,\dots,a_n \in k$
\\
\\
1.5
\\
Suponer que hay finitos polinomios mónicos irreducibles $F_1,\dots,F_n$, Sea $F=F_1 \dots F_n +1$, entonces $F$ es mónico irreducible o un producto de irreducibles, en el primer caso obtenemos una contradicción de lo que hemos supuesto, asi que consideramos el segundo caso
\\
\\
1.6
\\
Si $k$ es un campo finito con elementos $a_1,\dots,a_n$, el polinomio $f(X)=1+\prod_{i=1}^{n}(X-a_i)$ no posee raíz en $k$, por lo que no es algebraicamente cerrado.
\\
\\
1.7
%%%%%%%%%%%%%%%%%%%%%%%%%%%%%%%%%%%%%%%%%%%%%
\section*{1.2 Espacio Afín y Conjuntos Algebraicos}
\textbf{Definiciones y Teoremas:}
\\
\\
\textit{Definición 1:} Sea $k$ un campo cualquiera. Denotamos por $\A^{n}(k)$ al conjunto de $n$-tuplas con entradas en $k$. Denominamos este espacio como el \textit{($n$-)espacio afín}.
\\
\\
\textit{Definición 2:} Para $F \in k[X_1 , \dots, X_n]$ un punto $P \in \A{n}(k)$ se denomina un \textit{cero} de $F$ si $F(P)=F(a_1 , \dots , a_n)=0$. Si $F$ no es constante, el conjunto de ceros de $F$ se denomina \textit{hipersuperficie} definida por $F$ y denotada por $V(F)$.
\\
\\
(*) Generalmente, si $S$ es  un conjunto de polinomios en $k[X_1 , \dots , X_n]$, definimos $V(S)= \{P \in \A^{n}(k) \mid F(P)=0 \quad \forall F \in S$: $V(S)= \cap_{F \in S} V(F)$. Si $S= \{F_1, \dots, F_r \}$ escribimos $V(F_1 , \dots, F_r)$
\\
\\
\textit{Propiedades:} Se cumplen las siguientes propiedades:
\\
\\
(1) Si $I$ es el ideal en $k[X_1, \dots, X_n]$ generado por $S$, entonces $V(S)=V(I)$, por lo que cada conjunto algebraico es igual a $V(I)$ para algún ideal $I$.
\\
(2) Si $\{I_{\alpha} \}$ es cualquier colección de ideales, entonces $V(\bigcup_{\alpha} I_{\alpha})= \bigcap_{\alpha} V(I_{\alpha})$, por lo que cualquier colección de conjuntos algebraicos es un conjunto algebraico.
\\
(3) Si $I \subset J$, entonces $V(I) \supset V(J)$.
\\
(4) $V(FG)=V(F) \cup V(G)$ para polinomios $F,G$ cualquiera; $V(I) \cup V(J)=V( \{FG \mid F \in I, G \in J \})$; por lo que cualquier unión finita de conjuntos algebraicos es un conjunto algebraico.
\\
(5) $V(0)=\A^{n}(k)$; $V(1)=\emptyset$; $V(X_1 - a_1, \ots, X_n -a_n)=\{a_1, \dots,a_n \}$ para todo $a_i \in k$. Entonces, cualquier subconjunto finito de $\A^{n}(k)$ es un conjunto algebraico.
\\
\\
1.8
\\
Sea $\mathbb{A}^{1}(k)$, entonces estamos en $k[X_1]$, por lo que estamos en un PID. Sea el ideal $(F)$.
\\
\\
a) Si $F=0$, trivialmente, $V(0)=\mathbb{A}^{1}(k)$
\\
b) Si $F$ es no nulo, entonces por el teorema fundamental del álgebra, $f(X)$ posee a lo más $n$ soluciones diferentes, por lo que $V(p)=\{P \in \mathbb{A}^{n} | F(P) \}=\{ a_1,\dots,a_n\}$
\\
\\
1.9
\\
Sea el ideal $I_p$ de polinomios que se anulan en $p$, es decir $I_p=(X_1-a_1, \dots , X_n - a_n)$. Si $k$ es finito, un punto cualquiera $(a_1,\dots,a_n)$ puede ser de $k^n$ formas diferentes, por lo que cada subconjunto de $\mathbb{A}^{n}(k)$ es finito. Dado cualquier conjunto finito de puntos: $p_1,\dots,p_j$, si $I_{p_{i}}$ es el ideal de polinomios anulándose en $p_i$, entonces cada conjunto finito está definido como $V(I_1)\cup \dots \cup V(I_J)=v(I_1,\dots,I_j)$.
\\
\\
1.10
\\
Sea $\mathbb{Z} \subset \mathbb{R}$, sean la colección de conjuntos algebraicos de $\mathbb{A}^{1}(\mathbb{R})$, de la forma $V(X-k)$: $\{V(X-k) \}_{k \in \Z}$, esta  colección es contable, luego $\bigcup_{k \in \Z} V(X-k)$ no es algebraica. (Cada subconjunto debe ser finito)
\\
\\
1.11
\\
a) $\{ (t,t^{2},t^{3}) \in \A^{3}(k) \mid t \in k\}$. El conjunto corresponde a las soluciones simultáneas de $y=x^{2}$ y $z=x^3$, es decir el conjunto $V(y-x^{2}) \cap V(z-x^{3})=V(x^{2}-y,x^{3}-z)$
\\
\\
b) $\{ (\cos{t},\sin{t}) \in \A^{2}(\R) \mid t \in \R \}$. El conjunto corresponde a las soluciones de la ecuación $X^2 + Y^2 1$, entonces $V(X^2 +Y^2 -1)$ es conjunto algebraico.
\\
\\
c) $\{(r,\theta) \in \A^{2}(\R) \mid r=\sin{\theta} \}$. Por coordenadas polares, $x=r\cos{\theta}$, $y=r\sin{\theta}$, entonces $x=\sin{\theta} \cos{\theta}$; $y= \sin^{2}{\theta}$. Estas ecuaciones cumplen: $y=1-\cos^{2}(\theta) \Longrightarrow \cos{\theta}=\sqrt{1-y}$.
\\
\\
Entonces: $x=\sqrt{y}\sqrt{1-y} \Longrightarrow x^{2}=y(1-y) \Longrightarrow x^{2}=y-y^{2} \Longrightarrow V(x^{2}+y^{2}-y)$
\\
\\
1.12
\\
Sea $L: V(Y-(aX+b))$, entonces $Y=aX+b$, sea $F \in k[X,Y]$, por lo que $F(X,Y)$, reemplazando $Y=aX+b$, se obtiene $F(X,aX+b)$, por lo que tenemos una ecuación en una variable. Por el teorema fundamental del álgebra, hay a lo más $n$ soluciones diferentes.
\\
\\
1.13
\\
(a) $\{(x,y) \A^{2}(\R) \mid t \in k\}$
\\
\\
Se tiene que la curva $y=\sin{x}$ y la recta $y=0$ poseen infinitos puntos de intersección, esto no pasa con los conjuntos algebraicos. Ver que la unión de los conjuntos algebraicos descritos por las soluciones $a+2 \pi k$ no es conjunto algebraico. (ver ejercicio 1.10)
\\
\\
(b) $\{(z,w) \in A^{2}(\mathbb{C}) \mid |z|^2+|w|^2 =1\}$, donde $|x+iy|^{2}=x^2 +y^2$ para $x,y \in \R$
\\
\\
Sea $|z|^{2} + |w|^{2}=1$, finando $z=0$, entonces $|w|^{2}=1$, llevando $w$ a coordenadas polares se obtiene que $|r \operatorname{cis}\theta|^{2} =1$, entonces $|\operatorname{cis}{\theta}|=1/r^{2}$, entonces nos encontramos en un círculo, por lo que existen infinitas soluciones para $\theta$.
\\
\\
$\{(\cos{t},\sin{t},t) \in \A^{3}(\R) \mid t \in \R \}$
\\
\\
Se tiene que esta es la hélice, los puntos de la forma $(\cos{2k \pi}, \sin{2k \pi}, 2k \pi)$ poseen infinitos puntos de intersección con la hélice.
\\
\\
1.14
\\
Sea $n \geq 1$, como $k$ es algebraicamente cerrado, el polinomio $F(1,\dots,1,X_n)$ posee solamente soluciones finitas por el teorema fundamental del álgebra. por lo que el subconjunto:
\begin{equation*}
    \{(1,\dots , 1,a_n) \mid f(1,\dots,1,a_n) \neq 0 \} \subset \A^{k}-V(F)
\end{equation*}
es infinito, entonces $\A^{k}-V(F)$ lo es. Luego, sea $n \geq 2$, se cumple que para todo $a_1,a_2,\dots,a_{n-1} \in k$, el polinomio $F(a_1,a_2,\dots,a_{n-1},X_{n})$ posee al menos una raíz por el teorema fundamental del álgebra. por lo que $V(F)$ es finito. (no confundir los $a_1,\dots,a_{n-1}$, como valores fijos, notar el \textit{para todo} al comienzo.) 
\\
\\
1.15
\\
Sea $F \in I(V)$ tal que $F(a_1,\dots,a_n)=0$ y $G \in I(W)$ tal que $G(b_1,\dots,b_m)=0$. Se definen:
\begin{equation*}
\begin{aligned}
    &F=F'[X_1,\dots,X_n,X_{n+1},\dots,X_m] \\
    &F=G'[X_1,\dots,X_n,X_{n+1},\dots,X_m]
\end{aligned}
\end{equation*}
polinomios que les faltan algunas variables y que esto permite que aún sigan anulándose en los puntos anteriores; entonces:
\begin{equation*}
    V \times W = F(F' \cdot G')
\end{equation*}
\section*{1.3 El Ideal de un Conjunto de Puntos}
\textbf{Definiciones y Teoremas:}
\\
\\
\textit{Definición 1:} Para cualquier subconjunto $X$ de $\A^{n}(k)$, se tiene que el conjunto de polinomios que se anulan en $X$ forman un ideal en $k[X_1 , \dots, X_n]$; este se denota como $I(X)$ y se define como:
\begin{equation*}
    I(X)=\{F \in k[X_1, \dots, X_n] \mid F(a_1, \dots, a_n) =0 \quad \forall (a_1, \dots, a_n) \in X\}
\end{equation*}
\textit{Propiedades:}
\\
\\
(6) Si $X \subset Y$, entonces $I(X) \supset I(Y)$.
\\
(7) $I(\emptyset)=k[X_1 , \dots, X_n]$; $I(\A^{n}(k))=(0)$ si $k$ es un campo infinito; $I(\{(a_1, \dots, a_n) \})=(X_1-a_1, \dots, X_n-a_n)$ para $a_1, \dots, a_n \in k$.
\\
(8) $I(V(S)) \supset S$ para cualquier conjunto $S$ de polinomios; $V(I(X)) \supset X$ para cualquier conjunto $X$ de puntos.
\\
(9) $V(I(V(S)))=V(S)$ para cualquier conjunto $S$ de polinomios y $I(V(I(X)))=I(X)$ para cualquier conjunto $X$ de puntos. Entonces, si $V$ es un conjunto algebraico, $V=V(I(V))$ y si $I$ es el ideal de un conjunto algebraico, $I=I(V(I))$.
\\
\\
(*) Si $I=I(X)$ con $X$ conjunto algebraico, entonces si $F^n \in I$ para algún entero $n>0$, entonces $F \in I$.
\\
\\
\textit{Definición 2:} Se define el \textit{radical} de un ideal como:
\begin{equation*}
    \operatorname{Rad}{(I)}=\{a \in R \mid a^n \in I, \text{ para algún } n \in \mathbb{N} \}
\end{equation*}
Se tiene que $\operatorname{Rad}{(I)}$ es un ideal conteniendo $I$. Un ideal $I$ se dice $\textit{ideal radical}$ si $I=\operatorname{Rad}{(I)}$.
\\
\\
\textit{Propiedad 3:} $I(X)$ es ideal radical para cualquier $X \in \A^{n}(k)$.
\subsection*{Ejercicios:}
1.16
\\
$\Longrightarrow)$ Supongamos que $V=W$. Luego, sease $f \in I(V)$, entonces para $p \in V$, $f(p)=0$, pero $p \in W$ también, por lo que $f \in I(W)$. La igualdad se obtiene argumentando del mismo modo para $I(W)$.
\\
\\
$\Longleftarrow$) Supongamos que $I(V)=I(W)$. Sea $p \in V$, por suposición hay $f \in I(V)$ tal que $f(p)=0$, luego, $f \in I(W)$ también, entonces $p \in W$. Del mismo modo, la igualdad se obtiene cambiando en el argumento $I(V)$ por $I(W)$ de orden.
\\
\\
1.17
\\
(a) Sea $f \in I(V)$, se puede definir $F$ como $F(X_1, \dots,X_n)=\frac{f(X_1, \dots , X_n)}{f(P)}$
\\
\\
(b) Para cada punto, podemos definir un polinomio igual que en la parte (a) tal que solamente se anule en los demás puntos menos uno, formalmente:
\\
\\
Para todo elemento en $\{1,\dots,r \}$, se define el conjunto algebraico $V_i=\{P_1,\dots,P_{i-1},P_{i+1},\dots, P_r \}$, se define el polinomio que se anula en todos los conjuntos de la forma $V_j$: $G_{i}(V_i)$, pero que $G_{i}(P_i) \neq 0$, entonces:
\begin{equation*}
    F = \frac{G_i}{G_i(P_i)}
\end{equation*}
\\
\\
(c) Considerar $P_1,\dots, P_r$ como antes, queremos construir una función tal que $G_i(P_j)=a_ij$ para todo $i,j$, para esto nos aprovechamos de la sumatoria:
\begin{equation*}
    G_i = \sum_{j} a_{ij} F_j
\end{equation*}
Directamente podemos ver que $F_j(P_i)$ se anula si $j \neq i$, por lo que $G_i=a_{ij}$
\\
\\
1.18
\\
(a) Sea
\begin{equation*}
    (x+y)^{n+m} = \sum_{k=0}^{n+m} \binom{n+m}{k} x^{n+m-k} y^k
\end{equation*}
Se tiene que para todo $0 \leq k \leq m$, el grado que de $x$ es $\geq n$ y para todo $m \leq k \leq n+m$, todo grado $y$ es $\geq m$, por lo que cada monomio de la sumatoria pertenece a $I$ y a consecuencia, la sumatoria en sí.
\\
\\
(b) Sea $a+b \in R$, donde $a^n \in I$ y $b^m \in I$
\begin{itemize}
    \item $(a+b)^{n+m} \in I \Longrightarrow a+b \in \operatorname{Rad}{I}$
    \item $(ab)^{nm}=(a^{n})^m \cdot (b^{m})^{n} \in I \Longrightarrow ab \in \operatorname{Rad}{I}$
    \item Sea $x \in R$ y $a \in \operatorname{Rad}{I} \Longrightarrow (ax)^{n}=a^n x^n \in I \Longrightarrow ax \in \operatorname{Rad}{I}$
\end{itemize}
(c) Hay que probar $\operatorname{Rad}{I}=\operatorname{Rad}{(\operatorname{Rad}{I})}$
\\
\\
$\subset)$ Sea $x \in \operatorname{Rad}{\operatorname{Rad}{I}} \Longleftrightarrow x^n \in \operatorname{Rad}{I} \Longleftrightarrow (x^n)^n \in I \Longrightarrow x \in \operatorname{Rad}{I}$
\\
\\
$\supset)$ Sea $x \in \operatorname{Rad}{I} \Longleftrightarrow x^n \in I \Longleftrightarrow (x^n)^n \in I \Longrightarrow x^n \in \operatorname{Rad}{I} \Longrightarrow x \in \operatorname{Rad}{\operatorname{Rad}{I}}$
\\
\\
(d) Sea $P$ ideal primo.
\\
\\
Sea $x \in \operatorname{Rad}{P}$, por definición $x^n \in P$, como $P$ es primo $x \in P$ o $x^{n-1} \in P$, en el primer caso directamente demostramos que $x \in P$, en el segundo caso argumentamos de forma inductiva $x^{n-1}$ veces.
\\
\\
Sea $x^n \in P$, con $n \in \mathbb{N}$ minimal, entonces $x \in P$ o $x^{n-1} \in P$, en el primer caso directamente $x \in R$, en el segundo encontramos una contradicción ya que tomamos $n$ minimal.
\\
\\
1.19
\\
Para mostrar que es radical solo falta ver que es ideal primo, como el polinomio que describe $X^2 +1$ es irreducible en $\R[X]$, es ideal maximal y por lo tanto primo (su discriminante es negativo), y por lo tanto $(X^2 +1)=\operatorname{Rad}{(X^2+1)}$. Se tiene que $V(X^2+1)$ no describe ningún conjunto en $\A^{1}(\mathbb{R})$, ya que $\R$ no es algebraicamente cerrado.
\\
\\
1.20
\\
(a)
\\
$\Longrightarrow)$ Para $P \in V(I)$, todo $f \in I$ cumple que $f(P)=0$, pero entonces $f^n(x)=0$ también, es decir $f^{n} \in I$, esto es la definición de que $f \in \operatorname{Rad}{I}$. Concluimos que $P \in V(\operatorname{Rad}{I})$.
\\
\\
$\Longleftarrow)$ Sea $P \in V(\operatorname{Rad})$, entonces $\forall f \in \operatorname{Rad}{I}$ tal que $f(P)=0$, $f^n \in I$, en donde $f^n(P)=0$, por lo que $P \in V(I)$.
\\
\\
(b)
\\
Para $f \in \operatorname{Rad}{I}$, hay $n \in \mathbb{N}$ tal que $f^n \in I$, por lo que $f^n$ se anula en $V(I)$, es decir, para todo $x \in V(I)$, $f^n(x)=0 \Longleftrightarrow f(x)=0 \quad \forall x \in V(I)$. Es decir, $f(x) \in I(V(I))$. 
\\
\\
1.21
\\
(a) Sea $\varphi: k[x_1, \dots, X_n] \to k$ el mapa de evaluación. Entonces el kernel está descrito por aquellos polinomios en donde $f(a_1, \dots, a_n)=0 \Longleftrightarrow (X_i - a_i) \mid f(X_1, \dots , X_n) \quad \forall i \in \{ 1, \dots, n\}$. Entonces $\ker{\varphi} \subset (X_1 - a_1, \dots, X_n - a_n)$. La otra inclusión es directa, ya que todo polinomio en $I$ se cancela en el punto $(a_1, \dots, a_n)$. Es decir $\ker{\varphi}=(X_1 - a_1, \dots, X_n - a_n)$.
\\
\\
La imagen está dada por $k$. Ya que para todo $a \in k$ puedo definir el polinomio constante $a \in k[X_1, \dots, X_n]$. Por lo que $k[X_1, \dots , X_n]/I \cong k$.
\\
\\
(b) Para ver que el mapeo es isomorfismo solo basta ver que es la inversa de $\varphi$. O también ver que el kernel solo puede ser el $(0)$ y que para toda clase de equivalencia $a + I$ puedo definir $a \in k$.
\section*{1.4 Teorema de la base de Hilbert}
\subsection*{Teoremas y Definiciones:}
\textit{Teorema 1:} Cada conjunto algebraico es la intersección de un número finito de hiper superficies.
\\
\\
\textit{Teorema 2 (Teorema de la base de Hilbert):} Si $R$ es un anillo noetheriano, entonces $R[X_1, \dots, X_n]$ es noetheriano.
\\
\\
\textit{Corolrio 3:} $k[X_1, \dots, X_n]$ es noetheriano para todo campo $k$.
\end{document}