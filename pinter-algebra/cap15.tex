\documentclass{article}
\usepackage{amsmath}
\usepackage{amssymb}
\usepackage{graphicx}
\usepackage{hyperref}
\usepackage{mathrsfs}
\usepackage{dsfont}
\usepackage{hyperref}
\usepackage[margin=2cm]{geometry}

\newcommand{\R}{\mathbb{R}}
\renewcommand{\P}{\mathbb{P}}
\newcommand{\A}{\mathbb{A}}
\newcommand{\C}{\mathbb{C}}
\newcommand{\Z}{\mathbb{Z}}

\newcommand{\br}[1]{\left( #1 \right)}

\begin{document}
\subsection*{C. Relacionando Propiedades de $H$ con Propiedades de $G/H$}
Sea $G$ un grupo y $H$ un subgrupo normal de $G$. Probar:
\\
\\
1. Si $x^{2} \in H$ para cada $x \in G$, entonces cada elemento de $G/H$ es su propio inverso.
\\
\\
Si $x^{2} \in H$, entonces $x^{2}H=H$, esto es equivalente a decir que $xH=x^{-1}H$
\\
\\
2.AAAAAAA
\\
3. AAAAAAAAAAAAAAAAAAAA
\\
\\
4. Cada elemento de $G/H$ posee una raíz cuadrada si y solo si para todo $x \in G$ hay un $y \in G$ tal que $xy^{2} \in H$.
\\
\\
Si cada $Hx \in G/H$ posee una raíz cuadrada, hay $Hy$ tal que $Hx=Hy^{2}$, es decir $xy^{-2} \in H$. Las implicancias se leen en sus respectivas direcciones.
\\
\\
5. $G/H$ es cíclico si y solo si hay un elemento $a \in G$ con la siguiente propiedad: por cada $x \in G$, hay $n$ tal que $xa^{n} \in H$.
\\
\\
Si $G/H$ es cíclico, sease $\langle aH \rangle$ su generador, luego existe $xH \in \langle aH \rangle$ tal que $xH=a^{n}H$, para algún $n \in \mathbb{N}$, luego $x^{-1}a^{n} \in H$.
\\
\\
Si para todo $x \in G$, existe $n$ tal que $xa^{n} \in H$, etonces $xa^{n}H=H \Longleftrightarrow a^{n}H=x^{-1}H$, luego esta definición corresponde a grupo cíclico.
\\
\\
6. Si $G$ es un grupo abeliano, sea $H_p$ el conjunto de todos los $x \in G$ cual orden es una potencia de $p$. Probar que $H_p$ es un subgrupo de $G$. Probar que $G/H_p$ no posee elementos cual orden es ua potencia de $p$ (elemento no nulo.)
\\
\\
(i)Sea $a,b \in H_p$, luego $\operatorname{ord}{(a)}=p^{n}$ y $\operatorname{ord}{(b)}=p^{m}$, suponer que $m>n$. Luego $\operatorname{ord}{(ab)}=p^{m}$, $ab \in H_p$.
\\
(ii) Sea $a^{p^{n}}=e, luego e=a^{-p^{n}}=(a^{-1})^{p^{n}}$.
\\
(iii) Sea $e \in G$, luego $e^{p^{n}}=e$ para todo $n \in \mathbb{N}$.
\\
(iv) Supongamos que $a^p \in H_p$. Por definición, esto significa que el orden de $a^p$ es una potencia de $p$, es decir, existe un $n \in \mathbb{N}$ tal que:
\[
\operatorname{ord}(a^p) = p^n.
\]
Por lo tanto, se cumple que:
\[
(a^p)^{p^n} = e.
\]
Reescribiendo la expresión, obtenemos:
\[
a^{p^{n+1}} = e.
\]
Esto implica que el orden de $a$, denotado como $\operatorname{ord}(a)$, debe dividir $p^{n+1}$. Es decir, existe un entero $m$ tal que:
\[
\operatorname{ord}(a) = p^m.
\]
Dado que $p^m$ es una potencia de $p$, se concluye que $a \in H_p$, como queríamos demostrar.
\\
\\
7.
\\
(a) Si $G/H$ es abeliano, probar que $H$ contiene todos los conmutadores de $G$.
\\
\\
Si $G/H$ es abeliano, entonces $Hab=Hba$, luego $Haba^{-1}b^{-1}=H$, entonces $aba^{-1}b^{-1} \in H$
\\
\\
(b)
\\
\\
Sean $g,g'\in G$. Como $G/H$ es abeliano, para cualesquiera $g,g'\in G$ se tiene
\[
g\,g'H = g'H\,g.
\]
Esto implica que el conmutador
\[
[g,g'] = g\,g'\,g^{-1}\,g'^{-1} \in H.
\]
Dado que $H\subseteq K$, se tiene $[g,g']\in K$. Por lo tanto,
\[
g\,g' = [g,g']\,g'\,g,
\]
y al pasar al cociente obtenemos
\[
gK\;g'K = g'K\;gK.
\]
Concluimos que $G/K$ es abeliano.
\\
\\
Sea $k,k'\in K$. Dado que $k,k'\in G$ y $G/H$ es abeliano, el conmutador
\[
[k,k'] = k\,k'\,k^{-1}\,k'^{-1} \in H.
\]
Luego, en el cociente $K/H$ se tiene
\[
kH\,k'H = k'H\,kH.
\]
Por lo tanto, $K/H$ es abeliano.

\medskip

En consecuencia, si $G/H$ es abeliano, entonces tanto $G/K$ como $K/H$ son abelianos.
\subsection*{D. Propiedades de $G$ determinadas por propiedades de $G/H$ y $J$}
Hay propiedades de grupo donde si se cumplen en $G/H$ y en $H$, entonces se cumplen en $G$. Sea $G$ un grupo y $H$ un subgrupo normal de $G$. Probar:
\\
\\
1. Si cada elemento de $G/H$ posee orden finito y cada elemento de $H$ posee orden finito, entonces cada elemento de $G$ posee orden finito.
\\
\\
Sea $Hx$ con $x \in G$ en $G/H$, se tiene que hay $n \in \mathbb{N}$ tal que $Hx^{n}=H$, es decir $x^{n} \in H$, Luego, para elemento en $H$ posee orden finito, es decir, hay $m \in \mathbb{N}$ tal que $h^{m}=e$, luego esto se cumple para todo $x \in G$, ya que, $x^{n} \in H$, entonces $(x^{n})^m=e$. Luego cada $x \in G$ posee orden finito.
\\
\\
2. Si cada elemento de $G/H$ posee una raíz cuadrada y cada elemento de $H$ posee raíz cuadrada, entonces cada elemento de $G$ posee una raíz cuadrada.
\\
\\
Sea $Hx$ con $x \in G$ en $G/H$, existe $Hy$ tal que $Hx=Hy^{2}$ es decir $x(y^{-1})^{2} \in H$, luego cada $h \in H$ posee una raíz cuadrada también, por lo que $h=(h')^{2}$, luego como todo $x(y^{-1})^{2} \in H$ con $x,y \in G$, cada $g \in G$ posee raíz cuadrada.
\\
\\
3. Sea $p$ un número primo. Si $G/H$ y $H$ son \textit{p-grupos}, entonces $G$ es un \textit{p-grupo}.
\\
\\
Sea $Hx \in G/H$, se tiene que $Hx^{p^{n}}=H$, es decir $x^{p^{n}} \in H$, luego todo $h \in H$ cumple que $h^{p^{m}}=e$, como todo $x \in G$ cumple que $x^{p^{n}} \in H$, $(x^{p^{n}})^{p^{m}} = x^{p^{m+n}}$, luego $G$ es un p-grupo.
\\
\\
4. Si $G/H$ y $H$ son finitamente generados, entonces $G$ es finitamente generado.
\\
\\
Sea \( G/H \) finitamente generado por los cosets \( Hx_1, \dots, Hx_n \) y \( H \) finitamente generado por \( h_1, \dots, h_m \). Entonces, para cualquier \( g \in G \), se tiene que \( gH \) se puede escribir como un producto de los cosets \( Hx_i \), de modo que existe \( h \in H \) y enteros \( a_1, \dots, a_n \) tales que
\[
g = x_1^{a_1} x_2^{a_2} \cdots x_n^{a_n} h.
\]
Pero como \( h \) se puede expresar como un producto de \( h_1, \dots, h_m \), se sigue que \( g \) es producto de los elementos del conjunto finito 
\[
\{x_1, \dots, x_n, h_1, \dots, h_m\}.
\]
Por tanto, \( G \) es finitamente generado.
\subsection*{E. Orden de Elementos en Grupos Cocientes}
Sea $G$ un grupo y $H$ un subgrupo normal de $G$. Probar:
\\
\\
1. Por cada elemento $a \in G$, el orden del elemento $Ha$ en $G/H$ es un divisor del orden de $a \in G$.
\\
\\
Sea $\varphi:G \to G/H: a \mapsto Ha$ dado que $G/H$ es la imagen homomorfa de $G$. Luego se tiene que $\operatorname{ord}\varphi(a)\mid \operatorname{ord}{(a)}$ por ejercicio anterior, equivalentemente, $\operatorname{ord}(Ha) \mid \operatorname{ord}(a)$
\\
\\
2. Si $(G:H)=m$, el orden de cada elemento de $G/H$ es divisor de $m$.
\\
\\
Se tiene que $(G:H)$ es el orden de $G/H$, luego por teorema, para $Hg \in G/H$, $\operatorname{ord}(Hg) \mid |G/H|$.
\\
\\
3. Si $(G:H)=p$ con $p$ primo, entonces el orden de cada elemento $a \notin H$ en $G$ es un múltiplo de $p$.
\\
\\
$|G|=p\cdot |H|$, luego como para $g \notin H$, $\operatorname{ord}(g) \mid |G|$, es decir $\operatorname{ord}(g)\mid p \cdot |H|$, luego $\operatorname{ord}(g) \nmid |H|$, por lo que es múltiplo de $p$.
\\
\\
4. Si $G$ posee un subgrupo normal de índice $p$, donde $p$ es primo, entonces $G$ posee almenos un elemento de orden $p$.
\\
\\
Suponer que $G$ es finito, como $(G:H)=p$, entonces $G/H$ es cíclico, luego, como $G$ es finitio $p\mid |G|$, por lo que posee un elemento de orden $p$. Luego como es cíclico de orden $p$, cada elemento de $G$ es generador de $G$. Luego $G/H \cong \Z_{p}$.
\\
\\
5. Si $(G:H)=m$, entonces $a^{m} \in H$ para todo $a \in G$.
\\
\\
Se tiene que $\operatorname{ord}(Hx) \mid m$, luego $Ha^{m}=H$, luego $a^{m} \in H$.
\\
\\
6. En $\mathbb{Q}/\Z$, cada elemento posee orden finito.
\\
\\
Sea $\mathbb{Q}/\Z=\{\frac{m}{n} + \Z \mid (m,n)=1 \}$. Luego para $\frac{m}{n} + \Z$ en el conjunto se cumple que $n(\frac{m}{n} + \Z)=m+n\Z=m+\Z=\Z$.
\subsection*{F. Cociente de un Grupo por Su Centro}
El centro de un grupo $G$ es el subgrupo normal $C$ de $G$ consistiendo de todos los elementos de $G$ que conmutan con cada elemento de $G$. Suponer que el grupo cociente $G/C$ es grupo cíclico. Sease generado por $Ca \in G/C$. Probar:
\\
\\
1. Por cada $x \in G$, hay algún entero $m$ tal que $Cx=Ca^{m}$.
\\
\\
Como $G/C$ es cíclico, todo elemento $Cx$ con $x \in G$ es generado por una potencia de $Ca$, es decir, hay $m$ tal que $Cx=(Ca)^{m}=Ca^{m}$.
\\
\\
2. Por cada $x \in G$ hay algún entero $m$ tal que $x=ca^{m}$, donde $c \in C$.
\\
\\
Por el punto anterior, $Cx=Ca^{m}$, es decir, existen $c_1,c_2$ que: $c_1x=c_2a^{m} \Longleftrightarrow x=c_1^{-1}c_2a^{m}$, luego tomar $c=c_{1}^{-1}c_2$.
\\
\\
3. Para cualquier par de elementos $x,y \in G$, $xy=yx$.
\\
\\
Por punto anterior $x=ca^{m}$ y $y=c'a^{n}$, luego $xy=ca^{m}c'a_{n}$, como $c,c' \in C$ subgrupo normal, estos conmutan, es decir $ca^{m}c'a^{n}=c'a^{n}ca^{m}=yx$.
\\
\\
4. Luego Si $G/C$ es cíclico, se cumple que $xy=yx$, es decir, que $G$ es abeliano.
\\
\\
\subsection*{G. Usando la Ecuación de Clase para Determinar el Tamaño del Centro}
Sea $G$ un grupo finito. Un par de elementos $a,b \in G$ se dicen conjugados de uno y de otro si y solo si $a=xbx^{-1}$, para algún $x \in G$. Esto es una relación $a \sim b$ de equivalencia en $G$, la clase de equivalencia de cualquier elemento $a$ se dice \textit{clase de conjugación}. Entonces $G$ es particionado en clases de conjugación. (El tamaño de cada clase de conjugación divide a $|G|$.)
\\
\\
Sean $S_1, S_2,\dots,S_t$ las distintas clases de conjugación de $G$ y sea $k_1,k_2, \ots, k_t$ sus respectivos tamaños, entonces $|G|=k_1+k_2+\dots+k_t$ (Esta es la ecuación de clase de $G$)
\\
\\
Sea $G$ un grupo cual orden es una potencia de $p$, sease $|G|=p^{k}$. Sea $C$ el centro de $G$.
\\
\\
1. La clase de conjugación contiene a $a$ si y solo si $a \in C$.
\\
\\
Si $a \in [a]$, se tiene que $a \sim a$, es decir $a=xax^{-1}$ o equivalentemente $ax=xa$, por lo que $a \in C$.
\\
\\
2. Sea $c$ el orden de $C$. Entonces $|G|=c+k_s+k_{s+1}+\dots+k_t$, donde $k_{s},\dots,k_t$ son los tamaños de todas las clases de conjugación de elementos $x \notin C$.
\\
\\
Se tiene que cada elelemtno de $C$ es su propia clase de equivalencia, es decir para $a \in C$, $[a]$ contiene solamente $a$, por lo que en conjunto todas estas representan $|C|=c$ clases de equivalencia, luego tomando la ecuación de clase,
\begin{equation*}
    |G|=k_1+k_2+\dots+k_t
\end{equation*}
podemos sumar el tamaño de las clases de conjugación de elementos de $C$, (las cuales aportan indidividualmente $1$ en tamaño) obteniendo $c$ y dejar la ecuación como:
\begin{equation*}
    |G|=c+k_{s}+k_{s+1}+\dots+k_t
\end{equation*}
3. Por cada $i \in \{s,s+1,\dots,t \}$, $k_i$ es una potencia de $|G|$.
\\
\\
Vimos en $13-I6$ que el número de conjugados es un factor de $(G:C)$, luego el tamaño de cada clase de conjugación es un factor de $|G|$, luego $k_i=p^{j_i}$.
\\
\\
4. Despejando la ecuación $|G|=c+k_{s}+\dots+k_{t}$ para $c$, explicar por que $c$ es un múltiplo de $p$.
\\
\\
Se tiene que $c=|G|-(k_{s}+\dots+k_{t})$, luego $|G|=p^{k}$ y cada $k_i=p^{j_i}$, con $j_i \neq 0$, por lo que $p \mid c$.
\\
\\
\textit{Podemos concluir por la parte $4$ que $C$ debe contener más de un solo elemento $e$. De hecho $|C|$ es múltiplo de $p$}.
\\
\\
5. Probar: Si $|G|=p^{2}$, $G$ debe ser abeliano.
\\
\\
Si $|G|=p^{2}$, $|C|=p,p^{2}$ (no puede ser $1$ ya que debe ser múltiplo de $p$!). Si $|C|=p^{2}$, entonces $G=C$ y entonces todo elemento conmuta, asi que es directamente abeliano.
\\
\\
Si $|C|=p$, entonces $|G/C|=p$, por lo que $G/C$ es cíclico, luego por ejercicio $F$, $G$ es abeliano.
\\
\\
6. Probar que si $|G|=p^{2}$, entonces $G \cong \Z_{p^{2}}$ o $G \cong \Z_p \times \Z_p$.
\\
\\
Si $G$ es cíclico, hay un elemento $a$ que genera todo el grupo, es decir, posee orden $p^{2}$ , luego $\langle a \rangle=\Z_{p^{2}}$
\\
\\
Si no es cíclico, por teorema de Lagrange hay un subgrupo de orden $p$, ciclico, luego $|G-H|=p$, por lo que también es cíclico, luego tomando el mapeo $f:G \to \Z_p \times \Z_p: f(ab=(a,b)$ obtenemos un isomorfismo.
\\
\subsection*{Inducción en $|G|$: Un Ejemplo}
1. Si $\operatorname{ord}(a)=tp$ (para algún entero $t$), ¿Cuál es el elemento de $G$ que posee orden $p$?.
\\
\\
$\operatorname{ord}(a)=tp$, es decir $a^{tp}=e=(a^t)^p$. Luego $\operatorname{ord}(a^t)=p$
\end{document}