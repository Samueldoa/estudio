\documentclass{article}
\usepackage{amsmath}
\usepackage{amssymb}
\usepackage{graphicx}
\usepackage{hyperref}
\usepackage{mathrsfs}
\usepackage{dsfont}
\usepackage{hyperref}
\usepackage[margin=2cm]{geometry}

\newcommand{\R}{\mathbb{R}}
\renewcommand{\P}{\mathbb{P}}
\newcommand{\A}{\mathbb{A}}
\newcommand{\C}{\mathbb{C}}
\newcommand{\Z}{\mathbb{Z}}

\newcommand{\br}[1]{\left( #1 \right)}

\begin{document}
\subsection*{C. Propiedades Elementales de Homomorfismos}
Sea G,H y K grupos. Probar:
\\
\\
1. Si $f: G \to H$ y $g: H \to K$ son homomorfismos, entonces su composición $g \circ f: G \to K$ es un homomorfismo.
\\
\\
Sea $a,b \in G$, luego $(g \circ f)(ab)=g(f(ab))=g(f(a)f(b))=g(f(a))g(f(b))$
\\
\\
2. Si $f: G \to H$ es un homomorfismo con kernel $K$, entonces $f$ es inyectivo si y solo si $k=\{ e\}$
\\
\\
Sea $f$ inyectivo, entonces $f(a)=f(b)$ si $a=b$, luego, sea $x \in K$ tal que $f(x)=e$, luego como $f(e)=f(x)$, luego como $f$ es inyectivo se tiene que $x=e$. Por lo que todo elemento del kernel es el neutro. $K=\{e\}$.
\\
\\
Luego supongamos que $K=\{e\}$, sea $x,y$ tales que $f(x)=f(y) \Longleftrightarrow f(x)f^{-1}(y)=e=f(xy^{-1})$. Por lo que $xy^{-1} \in K$, pero por hipótesis $K=\{e\}$, es decir $xy^{-1}=e$, o equivalentemente, $x=y$.
\\
\\
3. Si $f:G \to H$ es homomorfismo y $K$ es cualquier subgrupo de $G$, entonces $f(K)=\{f(x)| x \in K\}$ es subgrupo de $H$.
\\
\\
Sea $e_{G} \in K$, luego $f(e_{G})=e_{K}$. Luego, sean $a,b \in K$, como $K$ es grupo, $ab \in K$, y como $f$ es homomorfismo, $f(ab)=f(a)f(b) \in f(K)$. Al ser $K$ grupo, para $a$ elemento, existe su inverso $a^{-1}$, luego $f(aa^{-1})=f(a)f(a^{-1})=f(a)f^{-1}(a) = e$. Luego $f^{-1}(a) \in f(K)$.
\\
\\
4. Si $f: G \to H$ es un homomorfismo y $J$ es cualquier subgrupo de $H$, entonces:
\begin{equation*}
    f^{-1}(J)=\{x \in G | f(x) \in J \}
\end{equation*}
es subgrupo de $G$. Además $\ker{f} \subseteq f^{-1}(J)$.
\\
\\
$f(e_G)=e_{H}$, luego en particular $e_{H} \in J$, por lo que $e_{G} \in f^{1}(J)$
\\
$f(a),f(b) \in J$, $f(a)f(b)=f(ab)$ entonces $ab \in f^{-1}(J)$
\\
$f(a),f^{-1}(a) \in J$, $f(a)f^{-1}(a)=f(a)f(a^{-1})=e$, luego $a^{-1} \in f^{-1}(J)$.
\\
\\
Luego, como $e_{H} \in J$, todo elemento $x \in \ker{f}$ pertenece a $f^{-1}(J)$, es decir $\ker{f} \subseteq f^{-1}(J)$.
\\
\\
5. Si $f: G \to H$ es homomorfismo con kernel $K$ y $J$ es subgrupo de $G$, sea $f_{J}$ la restricción de $f$ a $K$. Entonces $\ker{f_{J}}=J \cap K$.
\\
\\
Como $f_{J}(x)=f(x)$ para todo $x \in J$, se tiene que $\ker{f_{J}}=\{x \in J | f(x)=e_{H} \}$, luego ver que $\{x \in G |f(x)=e_{H}\}=\ker{f}=K$. Asi que todos los elementos $x \in J$ que satisfacen $f(x) = e_{H}$ son precisamente los elementos de $J$ que están en $K$. Luego $\ker{f_{J}}=J \cap K$.
\\
\\
6. Para cualquie grupo $G$, la función $f: G \to G$ definida por $f(x)=e$ es un homomorfismo.
\\
\\
Sea $a,b \in G$, luego $f(a)f(b)=e\cdot e=e=f(ab)$
\\
\\
7. Para cualquier grupo $G$, $\{e\}$ y $G$ son imagenes homomorfas de $G$.
\\
\\
a) $f(ab)=e=e\cdot e=f(a)f(b)$, luego para $e$ se puede definir siempre $g \in G$ tal que $f(g)=e$.
b) $f(ab)=ab=f(a)f(b)$, para $g \in G$ se puede definir $g \in G$ mismo elemento donde $f(g)=g$.
\\
\\
8. La función $f: G \to G$ definida por $f(x)=x^2$ es un homomorfismo si y solo si $G$ es abeliano.
\\
\\
Sea $a,b \in G$, luego $f(a)f(b)=a^{2}b^{2}=f(ab)=(ab)^{2}=abab$, luego $a^{2}b^{2}=abab$ si y solo si $ab=ba$. Se puede leer en ambas direcciones de la proposición.
\\
\\
9. Las funciones $f_{1}(x,y)=x$ y $f_{2}(x,y)=y$ desde $G \times H$ a $G$ y $H$ respectivamente son homomorfismos.
\\
\\
$f_{1}((x_1,y_1)(x_2,y_2))=f((x_1x_2,y_1y_2))=x_1x_2=f(x_1,y_1)f(x_2,y_2)$. El mismo procedimiento para $f_2$.
\subsection*{D. Propiedades Básicas de Subgrupos Normales}
Sea $G$ un grupo arbitrario.
\\
\\
1. Encontrar todos los subgrupos normales de $S_3$ y de $D_4$
\\
\\
$S_3$: $\{e\}$, $\{e,(123),(132)\}$
$D_4$: $\{e,r,r^{2},r^{3}\},\{e,r^{2},s,r^{2}s\}, \{e,r^{2},rs,r^{3}s\}$ y $\{e,r^{2}\}$
\\
\\
2. Cada subgrupo de un grupo abeliano es normal.
\\
\\
Sea $H$ subgrupo de un brupo abeliano, sea $h \in H$, luego $eh=h=gg^{-1}h=ghg^{-1} \in H$
\\
\\
3. El centro de cualquier grupo $G$ es un subgrupo normal de $G$.
\\
\\
Sea $C_{G}=\{a \in G | ax=xa \quad \forall x\in G \}$, luego sea $a \in G_{G}$, se tiene que $ax=xa \quad \forall x \in G$, pero esto es equivalente a $a=xax^{-1}$, es decir $C_{G}$ es normal con respecto a conjugados.
\\
\\
4. Sea $H$ un subgrupo de $G$. $H$ es normal si y solo si posee la siguiente propiedad: Para todo $a,b \in G$, $ab \in H$ si y solo si $ba \in H$.
\\
\\
$H$ es normal, entonces $xax^{-1} \in H$ para todo $x \in G$ y $a \in H$, sean $a,b \in H$ y $ab \in H$, como $H$ es normal, $babb^{-1}=ba \in H$, luego como $ba \in H,$ $abaa^{-1}=ab \in H$. Luego si $ab \in H \longleftrightarrow ba \in H$, entinces $ab=h_1$ y $ba=h_{2}$, luego $b=h_{1}a^{-1}$, por lo que $h_2=ah_1a^{-1}$, por lo que $H$ es normal.
\\
\\
5. Sea $H$ un subgrupo de $G$. $H$ es normal si y solo si $aH=Ha$ para todo $a \in G$
\\
\\
Si $H$ es normal, entonces $xh_1x^{-1}\in H$ para $h_1 \in H, x \in G$, es decir $xh_1x^{-1}=h_2 \Longleftrightarrow xh_1=h_2x \Longleftrightarrow xH=Hx$.
\\
\\
Si $aH=Ha$, entonces $ah_1=h_2a$, luego $h_2=ah_1a^{-1}$, es decir $ah_1a^{-1} \in H$.
\\
\\
6. Cualquier intersección de subgrupos normales es un subgrupo normal de $G$.
\\
\\
Sean $H,K$ subgrupos normales de $G$, sea $H \cap K$ y $a$ un elemento de este, luego como $H$ y $K$ son normales, $xax^{-1}\in H,K$, por lo que $xax^{-1} \in H \cap K$
\subsection*{E. Más Propiedades de Subgrupos Normales}
Sea $G$ un grupo y $H$ subgrupo de $G$. Probar:
\\
\\
1. Si $H$ posee índice $2$ en $G$, entonces $H$ es normal.
\\
\\
Como $(G:H)=2$, existen dos cosets de $H$, tanto izquierdos como derechos, luego estos son respectivamente, $H,aH$ y $H,Ha$ con $a \notin H$, como $eH=He$, entonces solo queda que $aH=Ha$.
\\
\\
2. Suponer que un elemento $a \in G$ posee orden $2$. Entonces $\langle a \rangle$ es subgrupo normal de $G$ si y solo si $a$ está en el centro de $G$.
\\
\\
Si $\langle a \rangle$ es subgrupo normal, entonces para $x \in G$, $xax^{-1} \in \langle a \rangle$, pero $\langle a \rangle = \{e,a \}$, como $xax^{-1}=e$ es imposible (ya que el orden es $2$), entonces $xax^{-1}=a$, es decir $xa=ax$ y $a \in C_{G}$.
\\
\\
3. Si $a$ es cualquier elemento de $G$, $\langle a \rangle$ es un subgrupo normal si y solo si $a$ posee la siguiente propiedad: para cualquier $x \in G$, hay un entero positivo $k$ tal que $xa=a^{k}x$.
\\
\\
Si $\langle a \rangle$ es subgrupo normal, entonces $xax^{-1} \in \langle a \rangle$, pero $\langle a \rangle = \{a^{i}| 1\leq i \leq \operatorname{ord}{(a)}\}$m es decur $xax^{-1}=a^{k} \Longleftrightarrow xa=a^{k}x$ para algún $k$.
\\
\\
Directamente si $xax^{-1}=a^{k}$ para algún $k$, como $a^{k} \in \langle a \rangle$, $xax^{-1}=\langle a \rangle$.
\\
\\
4. En un grupo $G$, un conmutador es cualquier producto de la forma $aba^{-1}b^{-1}$, donde $a$ y $b$ son elementos cualquiera de $G$. Si un subgrupo $H$ de $G$ contiene todos los conmutadores de $G$, entonces $H$ es normal.
\\
\\
Si $H=\{aba^{-1}b^{-1} | a,b \in G \}$, luego $g(aba^{-1}b^{-1})g^{-1}=(aha^{-1})(aga^{-1})(ah^{-1}a^{-1})(ag^{-1})a^{-1}=$
\\
$(aha^{-1})(aga^{-1})(aha^{-1})^{-1}(aga^{-1})^{-1} \in H$
\\
\\
5. Si $H$ y $K$ son subgrupos de $G$ y $K$ es normal, entonces $HK$ es un subgrupo de $G$. ($HK$ denota el conjunto de todos los productos $hk$ cuando $h$ para por $H$ y $k$ por $K$)
\\
\\
$e \in KH$ ya que $e=e\cdot e$
\\
Sea $h_1k_1$ y $h_2k_2 \in HK$, luego $h_1k_1h_2k_2$, como $K$ es normal $k_1h_2=h_2k_1'$, luego $h_1k_1h_2k_2=(h_1h_2)(k_1'k_2) \in HK$
\\
El inverso se ve directo al ser $H,K$ grupos.
\\
\\
6.
\subsection*{F. Homomorfismos y El Orden de Elementos}
Si $f: G \to H$ es un homomorfismo, probar:
\\
\\
1. Por cada elemento $a \in G$, el orden de $f(a)$ es divisor del orden de $a$.
\\
\\
Sea $a^{n}=e$ para algún $n \in \mathbb{N}$ minimal, luego $f(a^{n})=f(e)=e=f^{n}(a)$, luego, esto quiere decir que $m \mid n$ con $m$ el orden de $f(a)$.
\\
\\
2. El orden de cualquier elemento $b \neq e$ en el grando de $f$ es un divisor común de $|G|$ y $|H|$.
\\
\\
Se vió en $(1)$ que si $\operatorname{ord}{(f(a))}=m$ y $\operatorname{ord}{(a)}=n$, entonces $m \mid n$, luego como $n \mid |G|$, en particular $m \mid |G|$ a su vez $m \mid |H|$.
\\
\\
3. Si el rango de $f$ posee $n$ elementos entonces $x^{n} \in \ker{f}$ para todo $x \in G$.
\\
\\
Como $|\operatorname{ran}{f}|=n$, se tiene que $\operatorname{ord}{f(x)}\mid n$, por lo que $f^{n}(x)=f(x^{n})=e$. Luego $x^{n} \in \ker{f}$.
\\
\\
4. Sea $m$ un entero tal que $m$ y $|H|$ son coprimos. Para cualquier $x \in G$, si $x^{m} \in \ker{f}$, entonces $x \in \ker{f}$.
\\
\\
Sea $x^{m} \in \ker{f}$, entonces $f(x^{m})=f^{m}(x)=e$, luego, se sabe que $\operatorname{ord}{f(x)}|m$ y $\operatorname{ord}{f(x)}\mid |H|$, pero como son coprimos, $\operatorname{ord}{f(x)}=1$, luego $f(x)=e$ y $x \in \ker{f}$.
\\
\\
5. Sea el rango de $f$ de tamano $m$. Si $a \in G$ posee orden $n$, donde $m,n$ coprimos, entonces $a$ está en el kernel de $f$.
\\
\\
Sea $|\operatorname{ran}{f}|=m$, sea $a^{n}=e$, luego $f^{n}(a)=e$, luego se sabe que $\operatorname{ord}{f(a)} \mid n$, luego como $m,n$ son corpimos, no poseen factores primos, pero a la vez $\operatorname{ord}{f(a)} \mid m$, por lo que $\operatorname{ord}{f(a)}=1$, es decir $f(a)=e$ y $a \in \ker{f}$.
\\
\\
6. Sea $p$ primo. Si $\operatorname{ran}{f}$ posee un elemento de orden $p$, entonces $G$ posee un elemento de orden $p$.
\\
\\
Se tiene que $\operatorname{ord}{a} \mid |G|$, luego $p=\operatorname{ord}{f(a)} \mid \operatorname{ord}{a}$, es decir $p \mid |G|$, luego $G$ posee un elemento de orden $p$ (teorema de Cauchy).
\subsection*{G. Propiedades Preservadas bajo Homomorfismo}
Si $f: G \to H$ es un homomorfismo de $G$ a $H$ sobreyectivo. Probar:
\\
\\
1. Si $G$ es abeliano, entonces $H$ es abeliano.
\\
\\
$f(ab)=f(a)f(b)=f(ba)=f(b)f(a)$. Luego $f(a)f(b)=f(b)f(a)$.
\\
\\
2. Si $G$ es cíclico, entonces $H$ es cíclico.
\\
\\
Si $G$ es cíclico, entonces $G=\{a_{i} \mid i \in I \}$ con $I$ conjunto de enteros positivos contables, luego $f(a^{i})=f^{i}(a)$, por lo que $H$ posee generador $f(a)$.
\\
\\
3. Si cada elemento de $G$ posee orden finito, entonces cada elemento de $H$ posee orden finito.
\\
\\
Si para cada $a \in G$ existe $k \in \mathbb{N}$ tal que $a^{k}=e$, entonces $f(a^{k})=f(e)=e=f^{k}(a)$, luego cada elemento de $H$ posee orden finito.
\\
\\
4. Si cada elemento de $G$ es su propio inverso, cada elemento de $H$ es su propio inverso.
\\
\\
Si para todo $a \in G$, $a=a^{-1}$, luego $f(a)=f(a^{-1})=(f(a))^{-1}$.
\\
\\
5. Si cada elemento de $G$ pose una raíz cuadrada, entonces cada elemento de $H$ posee una raíz cuadrada.
\\
\\
Todo elemento $x$ posee raíz cuadrada, es decir existe $y \in G$ tal que $x=y^{2}$, luego $f(x)=f(y^{2})=f^{2}(y)$.
\\
\\
6. Si $G$ es finitamente generado, entonces $H$ es finitamente generado.
\\
\\
Sea $G=\langle g_1, \dots, g_{n} \rangle$, por lo que $a \in G$ se puede expresar como $a=g_{1}^{i_1}g_{2}^{i_{2}}\dots g_{n}^{i_{n}}$ donde cada $g_{i}$ está en el generador y los exponentes son $0,1 \text{ o }-1$, luego como $f$ es sobreyectiva, para cada $h \in H$ hay $g$ tal que $f(g)=h$, tomando $g$ como antes y considerando que $f$ es homomorfismo:
\begin{equation*}
    h=f(g)=f(g_{1}^{i_1}g_{2}^{i_{2}}\dots g_{n}^{i_{n}})=f(g_{1}^{i_1})f(g_{2}^{i_{2}})\dots f(g_{n}^{i_{n}})
\end{equation*}
Luego como $h$ fué arbitrario, todo elemento de $H$ se puede expresar de esta forma, por lo que $H = \langle f(g_{1}),f(g_{2}),\dots,f_{g_{n}} \rangle$.
\subsection*{H. Producto Directo Interno}
Si $G$ es un grupo cualquiera, sea $H,K$ subgrupos normales de este tal que $H \cap K=\{e\}$. Probar:
\\
\\
1. Si $h_1,h_2$ son dos elementos cualquiera de $H$ y $k_1,k_2$ elementos cualquiera de $K$.
\begin{equation*}
    h1k_1=h_2k_2 \text{ implica }h_1=h_2 \text{ y } k_{1}=k_{2}
\end{equation*}
Como $k_1k_1=h_2k_2 \Longleftrightarrow h_{2}^{-1}h_{1}k_{1}=k_{2} \Longleftrightarrow h_{2}^{-1}h_{1}=k_{2}k_{1}^{-1} \in H \cap K$, luego como $H \cap K = \{e\}$, $h_{2}^{-1}h_{1}=e$ y $k_{2}k_{1}^{-1}=e$, es decir, $h_{2}=k_{1}, k_{2}=k_{1}$.
\\
\\
2. Para cualquier $h \in H$ y $k \in K$, $hk=kh$.
\\
\\
Como $H$ y $K$ son normales, $H \cap K=\{e\}$ es normal, luego $hkh^{-1}k^{-1}=e$, es decir, $hk=kh$.
\\
\\
3. Ahora, hacer la suposición adicional que $G=HK$, eso es, cada $x \in G$ puede ser expresado como $x=hk$, para algún $h \in H$ y $k \in K$. Luego la función $\phi(h,k)=hk$ es un isomorfismo de $H \times K$ sobre $G$.
\\
\\
Sea $\phi(h_1,k_1)\phi(h_2,k_2)=h_1k_1h_2k_2=h_1h_2k_1h_2$ (por punto anterior conmutan), por lo que $\phi(h_1,k_1)\phi(h_2,k_2)=\phi(h_1h_2,k_1k_2)$.
\\
Sea $\phi(h_1,k_1)=\phi(h_2,k_2)$, es decir $h_1k_1=h_2k_2$, por punto anterior, $h_1=h_2$ y $k_1=k_2$, por lo que $(h_1,k_1)=(h_2,k_2)$, por lo que la inyectividad de cumple. Para ver la sobreyectividad ver que para todo producto $hk$ se puede definir $(h,k)$.
\subsection*{I. Subgrupos Conjugados}
Sea $H$ un subgrupo de $G$. Para cualquier $a \in G$, sea $aHa^{-1}=\{axa^{-1} \mid x \in H \}$. $aHa^{-1}$ se denomina un conjunto de $H$. Probar:
\\
\\
1. Por cada $a \in G$, $aHa^{-1}$ es subgrupo de $G$.
\\
\\
$e \in aHa^{-1}$ ya que $e \in H$, se tiene luego que $aea^{-1}=aa^{-1}=e$.
\\
Sea $aha^{-1} \in aHa^{-1}$, y $h^{-1} \in H$, luego se puede definir $ah^{-1}a^{-1}$, luego $(aha^{-1})(ah^{-1}a^{-1})=(ah)(h^{-1}a^{-1})=aa^{-1}=e$
\\
Sea $ah_{1}a^{-1}$ y $ah_{2}a^{-1} \in aHa^{-1}$, luego $(ah_{1}a^{-1})(ah_{2}a^{-1})=a(h_1h_2)a^{-1}$.
\\
\\
2. Por cada $a \in G$, $H \cong aHa^{-1}$.
\\
\\
Sea $\varphi:H \to aHa^{-1}: h \mapsto aha^{-1}$. Se tiene por punto anterior que $\varphi$ es homomorfismo.
\\
\\
\textit{Inyectividad}: $ah_{1}a^{-1}=ah_{2}a^{-1} \Longrightarrow h_{1}=h_{2}$
\\
\textit{Sobreyectividad}: Sea $aha^{-1}$, siempre se puede definir $h \in H$ tal que $\varphi(h)=aha^{-1}$.
\\
\\
3. $H$ es subgrupo normal de $G$ si y solo $H=aHa^{-1}$, para todo $a \in G$.
\\
\\
Si $H$ es subgrupo normal de $G$, entonces $aH=Ha$, es decir $ah_{1}=h_{2}a \Longleftrightarrow h_{2}=ah_{1}a^{-1}$, luego $H=aHa^{-1}$.
\\
\\
\textit{(*) En los ejercicios restantes, sea $G$ un grupo finito. Por el normalizador de $H$ nos referimos al conjunto $N(H)= \{ a \in G \mid axa^{-1} \in H \quad \forall x \in H \}$}
\\
\\
4. Si $a \in N(H)$, entonces $aHa^{-1}=H$ (recordar que $G$ es un grupo finito).
\\
\\
Sea $a \in N(H)$, entonces $aha^{-1} \in H$ para todo $h \in H$, por lo que $aHa^{-1} \subseteq H$, como $H \cong aHa^{-1}$, $|H| = |aHa^{-1}|$, por lo que $aHa^{-1}=H$.
\\
\\
5. $N(H)$ es subgrupo de $G$.
\\
\\
$e \in H$, luego $aea^{-1}=aa^{-1}=e$, por lo que $e \in N(H)$.
\\
Sea $h^{-1} \in H$, se tiene que $ah^{-1}a^{1} \in H$, ya que $(aha^{-1})(ah^{-1}h^{-1})=e$
\\
Sean $h_1,h_2 \in N(H)$, luego $ah_{1}a^{-1}$ y $ah_{2}a^{-1} \in H$, por lo que $(ah_1a^{-1})(ah_{2}a^{-1})=a(h_1h_2)a^{-1} \in H$.
\\
\\
6. $H \subseteq N(H)$. Además, $H$ es subgrupo normal de $N(H)$.
\\
\\
Sea $h \in H$, nortar que para $h' \in H$, $h'hh'^{-1} \in H$, luego $h \in N(H)$, por lo que $H \subseteq N(H)$. Para ver que $H$ es normal, tomar $a \in N(H)$ y $h \in H$, luego $aha^{-1} \in H$, luego es normal.
\\
\\
7. Para cualquier $a,b \in G$, $aHa^{-1}=bHb^{-1}$ si y solo si $b^{-1}a \in N$ si y solo si $aN=bN$.
\\
\\
Si $aHa^{-1}=bHb^{-1}$, $ah_{1}a^{-1}=bh_{2}b^{-1} \Longleftrightarrow b^{-1}ah_{1}a^{-1}b=(b^{-1}a)h_{1}(b^{-1}a)^{-1}=h_{2} \in H$, luego $b^{-1}a \in N$, es decor $aN=bN$. Ambas implicancias se leen en sus respectivas direcciones.
\\
\\
8. Hay una correspondencia uno-a-uno entre en conjunto de los conjugados de $H$ y el conjunto de cosets de $N$. (Entonces hay tantos conjugados de $H$ como cosets de $N$).
\\
\\
Sea $\varphi: \{aHa^{-1} \mid a \in G \} \to \{aN \mid a \in G \}: aHa^{-1} \mapsto aN$.
\\
\\
\textit{Inyectividad}: Si $aN=bN$, entonces $aHa^{-1}=bHb^{-1}$ (punto anterior)
\\
\textit{Sobreyectividad}: Luego por cada $aN$ se puede definir siempre $aHa^{-1}$ tal que $\varphi(aHa^{-1})=aN)$.
\\
\\
9. $H$ posee exactamente $(G:N)$ conjugados, en particular, el número de conjugados distintos de $H$ es divisor de $G$.
\\
\\
$\frac{|G|}{|H|}=(G:H)=|\{aN \mid a \in G\}|=|\{aHa^{-1} \mid a \in G \}|$, luego $|G|=|\{aHa^{-1} \mid a \in G \}|\cdot |H|$ y por lo tanto es divisor de $|G|$.
\\
\\
10. Lo mismo pero con la restricción $\varphi_{J}$.
\end{document}
