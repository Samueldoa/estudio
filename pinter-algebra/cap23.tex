\documentclass{article}
\usepackage{amsmath}
\usepackage{amssymb}
\usepackage{graphicx}
\usepackage{hyperref}
\usepackage{dsfont}
\usepackage[margin=2cm]{geometry}
\begin{document}
\section*{Capítulo 23}
\textbf{A. Resolviendo Congruencias Simples}
\\
\\
1.
\\
\\
a) $60x \equiv 12 \pmod{24}$. 
\begin{equation*}
    m = \frac{22}{\gcd{(60,24)}} = \frac{22}{12} = 2
\end{equation*}
b) $42x \equiv 24 \pmod{30}$
\begin{equation*}
    m = \frac{30}{\gcd{(42,30)}} = \frac{30}{6} = 5
\end{equation*}
c) $49x \equiv 30 \pmod{25}$
\begin{equation*}
    m = \frac{25}{\gcd{(49,25)}} = \frac{25}{1} = 25
\end{equation*}
d) $39x \equiv 14 \pmod{52}$
\begin{equation*}
    \text{No tiene solución, ya que: } \gcd{(39,52)} = 13, \text{ pero } 13 \nmid 14
\end{equation*}
e) $147x \equiv 47 \pmod{98}$
\begin{equation*}
    \text{No tiene solución, ya que: } \gcd{(98,147)=7,} \text{ pero } 7 \nmid 47
\end{equation*}
f) $39x \equiv 26 \pmod{52}$
\begin{equation*}
    m= \frac{26}{\gcd{(39,52)}} = \frac{26}{13} = 2
\end{equation*}
\\
2.
\\
\\
a) $12x \equiv 7 \pmod{25}$
\\
La ecuación no se puede reducir ya que $\gcd{(12,25)}=1$. $x \equiv 11 \pmod{25}$
\\
\\
b) $35x \equiv 8 \pmod{12}$.
\\
La ecuación no se puede reducir ya que $\gcd{(35,12)}=1$. $x \equiv 4 \pmod{12}$
\\
\\
c) $15x \equiv 9 \pmod{6}$
\\
Como $\gcd{(15,6)}=3$, la ecuación se reduce a $5x \equiv 3 \pmod{2}$. $x \equiv 1 \pmod{2}$
\\
\\
d) $42x \equiv 12 \pmod{30}$
\\
Como $\gcd{(42,30)}=6$, la ecuación se reduce a $7x \equiv 2 \pmod{5}$. $x \equiv 1 \pmod{5}$
\\
\\
e) $147x \equiv 49 \pmod{98}$
\\
Como $\gcd{(39,52)}=12 \Longrightarrow 3x \equiv 2 \pmod{4}$. $x \equiv 2 \pmod{4}$
\\
\\
f) $39x \equiv 25 \pmod{52}$
\\
Como $\gcd{(39,52)}=13 \Longrightarrow 3x \equiv 2 \pmod{4}$. $x \equiv 2 \pmod{4}$
\\
\\
3.
\\
a) Explicar por qué $2x^{2} \equiv 8 \pmod{10}$ posee las mismas soluciones que $x^{2} \equiv 4 \pmod{5}$.
\\
\\
Por teorema, existe una solución módulo $m$. En donde $m=10/\gcd{(2,10)}=10/2=5$. Con esto, la ecuación se puede reducir a $x^{2} \equiv 4 \pmod{5}$.
\\
\\
b) Explicar por qué $x \equiv 2 \pmod{5}$ y $x \equiv 3 \pmod{5}$ son todas las soluciones de $2x^{2} \equiv 8 \pmod{10}$
\\
\begin{equation*}
    \gcd(2,10) = 2 \Longrightarrow x^{2} \equiv 4 \pmod{5} \Longleftrightarrow x^{2} - 4 \equiv 0 \pmod{5} \Longleftrightarrow (x - 2)(x + 2) \equiv 0 \pmod{5}
\end{equation*}
\begin{equation*}
    \Longrightarrow x \equiv 2 \pmod{5} \text{ o } x \equiv 3 \pmod{5}
\end{equation*}
\\
\\
\\
4. Resolver las siguientes congruencias cuadráticas
\\
\\
a) $6x^{2} \equiv 9 \pmod{12}$.
\\
\\
$\gcd{(6,15)}=3 \Longrightarrow 2x^{2} \equiv 3 \pmod{5} \Longrightarrow x^{2} \equiv 4 \pmod{5} \Longleftrightarrow (x+2)(x-2) \equiv 0 \pmod{5} \Longrightarrow x \equiv 2 \pmod{5} \text{ o } x \equiv 3 \pmod{5}$
\\
\\
b) $60x^{2} \equiv 18 \pmod{24}$
\\
\\
$\gcd{(60,18)}=6 \Longrightarrow 10x^{2} \equiv 3 \pmod{4}$. La ecuación reducida no tiene solución ya que $\gcd{(10,4)} \nmid 3$
\\
\\
c) $30x^{2} \equiv 18 \pmod{24}$
\\
\\
$\gcd{(30,24)}=6 \Longrightarrow 5x^{2} \equiv 3 \pmod{4} \Longrightarrow x^{2} \equiv 3 \pmod{4}$. Esta ecuación no tiene soluciones en $\mathbb{Z}_{4}$
\\
\\
d) $4(x+1)^{2} \equiv 14 \pmod{10}$
\\
\\
$\gcd{(4,10)}=2 \Longrightarrow 2(x+1)^{2} \equiv 7 \pmod{5} \Longleftrightarrow (x+1)^{2} \equiv 1 \pmod{5} \Longleftrightarrow (x+1-1) \equiv 0 \pmod{5} \text{ o } (x+1+1) \equiv 0 \pmod{5} \Longrightarrow x \equiv 0 \pmod{5} \text{ o } x \equiv 3 \pmod{5}$
\\
\\
e) $4x^{2}-2x-2 \equiv 0 \pmod{6}$
\\
\\
La ecuación es equivalente a $4x^{2}+4x+4 \equiv 0 \pmod{6} \Longleftrightarrow 4(x^{2}+x+1) \equiv 0 \pmod{6} \Longleftrightarrow 2(x^{2}+x+1) \equiv 0 \pmod{3} \Longrightarrow x^{2}+x+1 \equiv 0 \pmod{3} \Longrightarrow x \equiv 1 \pmod{3}$
\\
\\
f) $3x^{2}-6x+6 \equiv 0 \pmod{15}$
\\
\\
$3(x^{2}-2x+2) \equiv 0 \pmod{15} \Longleftrightarrow 3(x^{2}-2x+1-1+2) \equiv 0 \pmod{15} \Longleftrightarrow 3\left( (x-1)^{2}+1\right) \equiv 0 \pmod{15} \Longleftrightarrow 3(x-1)^{2} \equiv 12 \pmod{15} \longrightarrow (x-1)^{2} \equiv 4 \pmod{5} \Longleftrightarrow x+1 \equiv 0 \pmod{5} \text{ o } x-3 \equiv 0 \pmod{5}$
\begin{equation*}
    x \equiv 4 \pmod{5} \text{ o } x \equiv 3 \pmod{5}
\end{equation*}
\\
\\
5. Resolver las siguientes congruencias
\\
\\
a) $x^{4} \equiv 4 \pmod{6}$
\\
\\
Equivalente a $x^{2}-2 \equiv 0 \pmod{6} \text{ o } x^{2}+2 \equiv 0 \pmod{6} \Longleftrightarrow x^{2} \equiv 2 \pmod{6} \text{ o } x^{2} \equiv 4 \pmod{6}$
\\
\\
$x^{2} \equiv 2 \pmod{6}$ no tiene solución. De la segunda ecuación se extrae que $x \equiv 2 \pmod{6} \text{ o } x \equiv 4 \pmod{6}$
\\
\\
b) $2(x-1)^{4} \equiv 0 \pmod{8}$.
\\
\\
$\gcd{(2,8)}=2 \Longrightarrow (x-1)^{4} \equiv 0 \pmod{4} \Longrightarrow 2^{2} \mid \left( (x-1)^{2}\right)^{2} \Longrightarrow (x-1) \equiv 0 \pmod{2} \Longleftrightarrow x \equiv 1 \pmod{2}$
\\
\\
c) $x^{3}+3x^{2}+3x+1 \equiv 0 \pmod{8}$
\\
\\
$(x+1)^{3} \equiv 0 \pmod{8} \Longrightarrow 2^{3} \mid (x+1)^{3} \Longrightarrow 2 \mid (x+1) \Longleftrightarrow x+1 \equiv 0 \pmod{2} \Longleftrightarrow x \equiv 1 \pmod{2}$
\\
\\
d) $x^{4}+2x^{2}+1 \equiv 4 \pmod{5}$
\\
\\
$(x^{2}+1)^{2} \equiv 4 \pmod{5} \Longleftrightarrow (x^{2}-1)(x^{2}+3) \equiv 0 \pmod{5} \Longleftrightarrow x^{2} \equiv 1 \pmod{5} \text{ o } x^{2} \equiv 2 \pmod{5}$
\\
\\
La ecuación $\bar{x}^{2} = \bar{2}$ no posee solución en $\mathbb{Z}_{5}$. De la primera ecuación se obtiene que $x \equiv 1 \pmod{5} \text{ o } x \equiv 4 \pmod{5}$
\\
\\
6. Resolver las siguientes ecuaciones diofantinas (Si no tienen solución, argumentar).
\\
\\
a) $14x+15y=11$
\begin{equation*}
\begin{aligned}
    &14x \equiv 11 \pmod{15} \Longrightarrow x \equiv 4 \pmod{15} \\
    &15y \equiv 11 \pmod{14} \Longrightarrow x \equiv 11 \pmod{14}
\end{aligned}
\end{equation*}
b) $4x+5y=1$
\begin{equation*}
\begin{aligned}
    &4x \equiv 1 \pmod{5} \Longrightarrow x \equiv 4 \pmod{5} \\
    &5y \equiv 1 \pmod{4} \Longrightarrow x \equiv 1 \pmod{4}
\end{aligned}
\end{equation*}
c) $21x + 10y =9$
\begin{equation*}
\begin{aligned}
    &21x \equiv 9 \pmod{10} \Longrightarrow x \equiv 9 \pmod{10} \\
    &10y \equiv 9 \pmod{21} \Longrightarrow x \equiv 3 \pmod{21}
\end{aligned} 
\end{equation*}
d) $30x^{2}+24y=18$
\begin{equation*}
\begin{aligned}
    &24y \equiv 18 \pmod{30} \Longrightarrow 4y \equiv 3 \pmod{5} \Longrightarrow x \equiv 2 \pmod{5} \\
    &30x^{2} \equiv 18 \pmod{24} \Longrightarrow 5x^{2} \equiv 3 \pmod{4} \Longleftarrow x^{2} \equiv 3 \pmod{4} \Longleftrightarrow \bar{x}^{2} = \bar{3} \text{ no tiene solución en } \mathbb{Z}_{4}
\end{aligned}
\end{equation*}
\\
\textbf{C. Propiedades Elementales de Congruencias}
\\
\\
Probar las siguientes propiedades para enteros $a,b,c,d$ y enteros positivos $m$ y $n$.
\\
\\
1. Si $a \equiv b \pmod{n}$ y $b \equiv c \pmod{n}$, entonces $a \equiv c \pmod{n}$
\\
\\
$nk_{1}=(a-b)$ y $nk_{2}(b-c)$, entonces $nk_{1}+nk_{2}=(a-b)+(b-c) \Longrightarrow  n \mid (a-c)$
\\
\\
2. Si $a \equiv b \pmod{n}$, entonces $a+c \equiv b+c \pmod{n}$
\\
\\
$nk=(a-c) \Longleftrightarrow nk=(a+c-(c+b)) \Longrightarrow a+c \equiv b+c \pmod{n}$
\\
\\
3. Si $a \equiv b \pmod{n}$, entonces $ac \equiv bc \pmod{n}$
\\
\\
$nk=(a-b) \Longleftrightarrow cnk=c(a-b) \Longrightarrow n \mid c(a-b) \Longrightarrow n \mid ac-bc$
\\
\\
4. $a \equiv b \pmod{1}$
\\
\\
Directamente, $1$ divide todo entero: $1 \mid a-b$
\\
\\
5. Si $ab \equiv 0 \pmod{p}$, donde $p$ es primo, entonces $a \equiv 0 \pmod{n}$ o $b \equiv 0 \pmod{n}$
\\
\\
$p \mid ab$, por lema de Euclides $p \mid a$ o $p \mid b$, es decir $a \equiv 0 \pmod{p}$ o $b \equiv 0 \pmod{p}$
\\
\\
6. Si $a^{2} \equiv b^{2} \pmod{p}$, donde $p$ es primo, entonces $a \equiv \pm b \pmod{p}$
\\
\\
$p \mid (a^{2}-b^{2}) \Longleftrightarrow p \mid (a-b)(a+b)$, por lema de Euclides, $p \mid (a-b)$ o $p \mid (a+b)$, entonces $a \equiv \pm b \pmod{p}$
\\
\\
7. Si $a \equiv b \pmod{m}$m entonces $a+km \equiv b \pmod$
\\
\\
$m \mid (b-a) \Longrightarrow mk=(b-a) \Longleftrightarrow mk+a-b = 0 \Longleftrightarrow mk+a \equiv b \pmod{m}$
\\
\\
8. Si $ac \equiv bc \pmod{n}$ y $\gcd{(c,n)}=1$, entonces $a \equiv b \pmod{n}$
\\
\\
\begin{equation*}
\begin{aligned}
    &ac-bc=nk \\
    &c(a-b) = kn
\end{aligned}
\end{equation*}
Como $\gcd{c,n}=1$, entonces $n \mid (a-b)$, es decir $a \equiv b \pmod{n}$
\\
\\
9. Si $a \equiv b \pmod{n}$, entonces $a \equiv b \pmod{m}$, donde $m$ es un factor de $n$.
\\
\\
$n \mid (a-b) \Longleftrightarrow nk = (a-b) \Longleftrightarrow mqk=(a-b) \Longleftrightarrow m \mid (a-b) \Longleftrightarrow a \equiv b \pmod{m}$
\\
\\
\textbf{E. Consecuencias del Teorema de Fermat}
\end{document}