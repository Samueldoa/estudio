\documentclass{article}
\usepackage{amsmath}
\usepackage{amssymb}
\usepackage{graphicx}
\usepackage{hyperref}
\usepackage{mathrsfs}
\usepackage{dsfont}
\usepackage{hyperref}
\usepackage[margin=2cm]{geometry}

\newcommand{\R}{\mathbb{R}}
\renewcommand{\P}{\mathbb{P}}
\newcommand{\A}{\mathbb{A}}
\newcommand{\C}{\mathbb{C}}
\newcommand{\Z}{\mathbb{Z}}

\newcommand{\br}[1]{\left( #1 \right)}

\begin{document}
\title{\textbf{Capítulo 10 - Orden de Elementos de un Grupo}}
\date{}
\maketitle
\textbf{A. Leyes de Exponentes}
\\
\
1. Probar $a^{m}a^{n}=a^{m+n}$
\begin{equation*}
\begin{aligned}
    &(a) \quad m = 0: a^{m}a^{n}=a^{0}a^{n}=a^{n}=a^{0+n} \\
    &(b) \quad m<0, n>0: a^{m}a^{n}=(a^{-m})^{-1}(a^{n})=\underbrace{(a \cdots a)^{-1}}_{-m \text{ veces}} \underbrace{(a \cdots a)}_{n \text{ veces}} = \underbrace{(a^{-1} \cdots a^{-1})}_{-m \text{ veces}} = \underbrace{(a \cdots a)}_{n-(-m) \text{ veces}} \\
    &(c) \quad m<0, n<0: (a^{-m})^{-1} (a_{-n})^{-1} = \underbrace{(a \cdots a)^{-1}}_{-m \text{ veces}} \underbrace{(a \cdots a)^{-1}}_{-n \text{ veces}} = \underbrace{a^{-1} \cdots a^{-1}}_{-n-m \text{ veces}}=(a^{-1})^{-n-m}=a^{m+n}
\end{aligned}
\end{equation*}
2. Probar que $(a^{m})^{n}=a^{mn}$ en los siguientes casos:
\begin{equation*}
\begin{aligned}
    &(a) \quad m=0: (a^{m})^n=(a^0)^n=e^n = e^0 = a^{0n} \quad (\text{no}) \\
    &(b) \quad n=0: (a^m)^0=\underbrace{(a \cdots a)^{0}}_{m \text{ veces}}= \underbrace{a^{0} \cdots a^{0}}_{m \text{ veces}} = a^{m \cdot 0} \\
    &(c) \quad m<0, n>0: (a^{m})^{n}=\left((a^{-m})^{-1} \right)^n = \underbrace{(a^{-m})^{-1} \cdots (a^{-m})^{-1}}_{n \text{ veces}} = 
    \underbrace{\underbrace{(a^{-1} \cdots a^{-1})}_{-m \text{ veces}} \cdots \underbrace{(a^{-1} \cdots a^{-1})}_{-m \text{ veces}}}_{n \text{ veces}}=(a^{-1})^{-mn}=a^{mn} \\
    &(d) \quad m>0, n<0: (a^{m})^n = \left((a^m)^{-n}\right)^{-1} = \underbrace{(a^{m} \cdots a^{m})^{-1}}_{-n \text{ veces}} = \underbrace{(a \cdots a)^{-1}}_{-nm \text{ veces}} = a^{-1} \cdots a^{-1} = (a^{-1})^{-nm}=a^{nm}
\end{aligned}
\end{equation*}
\begin{equation*}
\begin{aligned}
    (e) \quad m<0, n<0: (a^m)^n&=\left( (((a^{-m}))^{-1})^{-n}\right)^{-1}= ( \underbrace{(a^{-m})^{-1} \cdots (a^{-m})^{-1})}_{-n \text{ veces}})^{-1} \\
    &= (\underbrace{\underbrace{(a \cdots a)^{-1}}_{-m \text{ veces}} \cdots (a \cdots a)^{-1}}_{-n \text{ veces}})^{-1} = \underbrace{(a^{-1} \cdots a^{-1})^{-1}}_{mn \text{ veces}} = ((a^{-1})^{nm})^{-1}=(a^{-nm})^{-1}=a^{nm}
\end{aligned}
\end{equation*}
3. Probar que $(a^{n})^{-1}=a^{-n}$
\begin{equation*}
\begin{aligned}
    (a) \quad n=0: (a^{n})^{-1} = (a^{0})^{-1}=a^{-0}=a^{0} \\
    (b) \quad n<0: (a^{n})^{-1} = ((a^{-n})^{-1})^{-1} = (a^{-1} \cdots a^{-1})^{-1} = a^{-1} \cdots a^{-1} = a^{-n}
\end{aligned}
\end{equation*}
\textbf{B. Ejemplo de Ordenes de Elementos}
\\
\\
1. Orden de $10$ en $\mathbb{Z}_{25}$
\begin{equation*}
    5 \cdot 10 = 0 \pmod{\mathbb{Z}_{25}}
\end{equation*}
2. Orden de $6$ en $\mathbb{Z}_{16}$
\begin{equation*}
    8 \cdot 6 = 0 \pmod{\mathbb{Z}_{16}}
\end{equation*}
3. Orden de $f = \begin{pmatrix}
1 & 2 & 3 & 4 & 5 & 6\\
6 & 1 & 3 & 2 & 5 & 4
\end{pmatrix}$
\begin{equation*}
\begin{aligned}
    &f^{2} = \begin{pmatrix}         
1 & 2 & 3 & 4 & 5 & 6\\
4 & 6 & 3 & 1 & 5 & 2
\end{pmatrix} \quad \quad f^{3}=\begin{pmatrix}
1 & 2 & 3 & 4 & 5 & 6\\
2 & 4 & 3 & 6 & 5 & 1
\end{pmatrix} \quad \quad
&f^{4}= \begin{pmatrix}
1 & 2 & 3 & 4 & 5 & 6\\
1 & 2 & 3 & 4 & 5 & 6
\end{pmatrix}
\end{aligned}
\end{equation*}
4. Orden de $1$ en $\mathbb{R}^{*}$ y en $\mathbb{1}$.
\\
\\
Uhhhhhhhhhhhhhhhhhhhhhhhhhhhhhhhhhhhh 
\\
\\
5. Orden de $f(x)=\frac{2}{2-x}$
\begin{equation*}
\begin{aligned}
    f^{2}=\frac{2-x}{1-x} \quad \quad f^{3}=\frac{2x-2}{x} \quad \quad f^{4}= \frac{2}{2-x}
\end{aligned}
\end{equation*}
6. Encontrar un grupo infinito tal que tenga un elemento de orden finito:
\\
\\
Sea el cociente $\mathbb{Q}/\mathbb{Z}$, entonces para $p/q + \mathbb{Z}$, el elemento posee orden $q$: $q \cdot (p/q)+\mathbb{Z} = p + \mathbb{Z} = 0+\mathbb{Z}$
\\
\\
7. En $\mathbb{Z}_{24}$ listar todos los elementos de orden $2$, $3$, $4$, $6$.
\begin{equation*}
\begin{aligned}
    &\text{Orden } 2: \quad 12 \\
    &\text{Orden } 3: \quad 8,16 \\
    &\text{Orden } 4: \quad 6,18 \\
    &\text{Orden } 6: \quad 4,8,16,20
\end{aligned}
\end{equation*}
\textbf{C. Propiedades Elementales del Orden}
\\
Sean $a$, $b$ y $c$ elementos de un grupo $G$.
\\
\\
1. $\operatorname{ord}(a)=1$ ssi $a=e$
\\
$\Longrightarrow)$ $a^{1}=e \Longleftrightarrow a = e$.
\\
$\Longleftarrow)$ $a=e \Longleftrightarrow a^{1}=e$
\\
\\
2. Si $\operatorname{ord}(a)=n$, entonces $a^{n-r}=(a^{r})^{-1}$
\\
$\longrightarrow)$ $a^{n-r}=a^{n} \cdot a^{-r}= e \cdot a^{-r}= (a^{r})^{-1}$
\\
\\
3. Si $a^{k}=e$ donde $k$ es impar, entonces el orden de $a$ es impar.
\\
\\
Si $a^{k}=e$, entonces $k$ es múltiplo de $\operatorname{ord}(a)=n$, es decir $k=n \cdot q$, tanto $n$ como $q$ no pueden ser pares, pues tendríamos una contradición por la imparidad de $n$.
\\
\\
4. $\operatorname{ord}(a)=\operatorname{ord}(bab^{-1})$
\\
\\
Sea $a^{n}=e$, luego $(bab^{-1})^{n}=(ban^{-1})(bab^{-1}) \cdots (bab^{-1})=ba^{n}b^{-1}=beb^{-1}=bb^{-1}=e$
\\
\\
5. El orden de $a^{-1}$ es el mismo que el orden de $a$
\\
\\
Sea $n$ el orden de $a$, luego $(a^{-1})^{n}=a^{-n}=(a^{n})^{-1}=e^{-1}=e$
\\
\\
6. El orden de $ab$ es el mismo que el de $ba$.
\\
\\
$(ba)^{n}=baba \cdots ab=e  \Longleftrightarrow \underbrace{(baba \cdots bab)}_{x}a = e$, es decir $a$ es el inverso de $x$, entonces, $a(baba \cdots bab) = e = (ab)^{n}$
\\
\\
7. $\operatorname{ord}(abc)=\operatorname{ord}(cab)=\operatorname{ord}(bca)$
\begin{equation*}
\begin{aligned}
    (abc)^{n}=(ab)c \cdot (ab)c \cdots (ab)c = e &\Longleftrightarrow c(ab)c(ab) \cdots (ab)c(ab) = e \\
    &\Longleftrightarrow (cab)^{n}=e \\
    &\Longleftrightarrow (ca)b \cdot (ca)b \cdots (ca)b = e \\
    &\Longleftrightarrow b(ca)b(ca)b \cdots (ca) = e \\
    &\Longleftrightarrow (bca)^{n}=e
\end{aligned}
\end{equation*}
8. Sea $x=a_1 a_2 \cdots a_n$ y sea $y$ el producto de los mismos factores permutados de forma cíclica.
\\
\\
Argumentamos por inducción
\\
\\
\textbf{D. Más propiedades del orden}
\\
Sea $a$ un elemento de orden finito en un grupo $G$.
\\
\\
1. Si $a^{p}=e$, con $p$ primo, entonces $a$ posee orden $p$ ($a \neq e$)
\\
\\
Si $a$ tuviese orden no $p$, se tendría que $p$ posee como factor el orden, $p = \operatorname{ord}(a) \cdot q$, pero los únicos divisores de $p$ son $p$ y $1$. No puede ser $1$ ya que por hipótesis $a \neq e$, entonces $\operatorname{ord}(a)=p$
\\
\\
2. El orden de $a^{k}$ es un divisor (factor) del orden de $a$.
\\
\\
Sea $n$ el orden de $a^{k}$, es decir $(a^{k})^{n}=e$ si y solo si $a^{kn}=e$, luego, $kn=\operatorname{ord}(a) \cdot q + r$: $a^{\operatorname{ord}(a) \cdot q + r}=e \cdot a^{r}=e$, entonces $r=0$, por lo que $kn=\operatorname{ord}(a) \cdot q \Longleftrightarrow k \operatorname{ord}(a^{k})= \operatorname{ord}(a) \cdot q$, entonces $\operatorname{ord}(a)=\frac{k}{q} \cdot \operatorname{ord}(a^{k})$
\\
\\
3. Si $\operatorname{ord}(a)=km$, entonces $\operatorname{a^{k}}=m$
\\
\\
$a^{km}=e \Longleftrightarrow (a^{k})^{m}=e \Longrightarrow \operatorname{ord}(a^{k})=m$
\\
\\
4.
\\
\\
5. Si $a$ posee orden $n$ y $a^{r}=a^{s}$, entonces $n$ es factor de $r-s$:
\\
\\
$a^{r}=a^{s}$ si y solo si $a^{r-s}=e$, entonces por teorema $r-s=n \cdot q$
\\
\\
6. Si $a$ es el único elemento de orden $j$ en $G$, entonces $a$ está en el centro de $G$.
\\
\\
Por ejercicio anterior, $\operatorname{ord}(a)=\operatorname{ord}(bab^{-1})$, pero $a$ es el único elemento con este orden, por lo que $a=bab^{-1} \Longleftrightarrow ab = ba$, por lo que $a$ está en el centro.
\\
\\
7. Si el orden de $a$ no es un múltiplo de $m$, el orden de $a^{k}$ no es múltiplo de $m$
\\
\\
Sea $\operatorname{ord}(a) \neq m \cdot q$, se tiene que $\operatorname{ord}(a^{k}) \mid \operatorname{ord}(a)$, entonces $k \cdot \operatorname{a^{k}}=\operatorname{ord}(a) \neq m \cdot q$.
\\
\\
8. Si $\operatorname{a}=mk$ y $a^{rk}=e$, entonces $r$ es múltiplo de $m$.
\\
\\
$rk=mr \cdot q + r$, Luego, $a^{mk \cdot q +r}=a^{mk \cdot q} \cdot a^{r}=e \cdot a^{r}=e$, entonces $r=0$, por lo que $rk=mkq$ si y solo si $r=mq$.
\\
\\
\textbf{F. Orden de Potencias de Elementos}
\\
Sea $a$ un elemento de orden $12$ en un grupo $G$.
\\
\\
1. ¿Cuál es el entero positivo $k$ más pequeño tal que $a^{8k}=e$?
\\
\\
$8k = 12q \Longleftrightarrow 8k \equiv 0 \pmod{12} \Longleftrightarrow k=3$
\\
\\
2. ¿Cuál es el orden de $a^{8}$?
\\
\\
Por punto anterior $k=3$
\\
\\
3. ¿Cuáles son los órdenes de $a^{9},a^{10},a^{5}?$
\begin{itemize}
    \item $9k =12q \Longleftrightarrow 9k \equiv 0 \pmod{12} \Longleftrightarrow k=4$ 
    \item $10k \equiv 0 \pmod{12} \Longleftrightarrow k = 6$
    \item $5k \equiv 0 \pmod{12} \Longleftrightarrow k =12$
\end{itemize}
4. ¿Cuál de las potencias de $a$ poseen el mismo orden que $a$?
\\
\\
$(\operatorname{ord}(a^{k}))k=12q=\operatorname{ord}(a)q \Longleftrightarrow \operatorname{ord}(a^{k})k \equiv 0 \pmod{12} \Longleftrightarrow \operatorname{ord}(a^{k}) \equiv 0 \pmod{m}$ donde $m = \frac{12}{\gcd{(12,k)}}$, como queremos que el orden sea $12$ buscamos los $k$ tal que sean coprimos con $12$, es decir $k=1,5,7,11$.
\\
\\
5. Sea $a$ un elemento de orden $m$ en cualquier grupo $G$. ¿Cuál es el orden de $a^{k}$? (viendo los ejemplos anteriores y generalizando, no hay que probar nada)
\\
\\
\begin{equation*}
    \operatorname{ord}(a^{k})=\frac{\operatorname{ord}(a)}{\gcd{(\operatorname{ord}(a),k)}}
\end{equation*}
6. Sea $a$ un elemento de orden $m$ en cualquier grupo $G$. ¿Para qué valores de $k$ es $\operatorname{ord}(a^{k})=m$
\\
\\
Para aquellos $k$ coprimos al orden de $a$.
\\
\\
\textbf{G. Relación entre $\operatorname{ord}(a)$ y $\operatorname{ord}(a^k)$}
\\
Sea $a$ un elemento de orden $n$ en un grupo $G$.
\\
\\
1. Probar que si $m$ y $n$ sn coprimos, entonces $a^{m}$ posee orden $n$.
\\
\\
Por parte anterior, podemos ver que $\operatorname{ord}(a^{k})=\frac{\operatorname{ord}(a)}{\gcd{(\operatorname{ord}(a),k)}}=\frac{n}{\gcd{(m,n)}}=n$
\\
\\
Por otro lado, usando el teorema anterior, $mk=nq$, entonces $n$ es factor de $mk$, pero como $m$ es coprimo con $n$, no comparten factores primos, entonces $n$ es factor de $k$, es decir $a^{mnk_{0}}=e$
\\
\\
2. Probar que si $a^{m}$ posee orden $n$, entonces $m$ y $n$ son coprimos.
\begin{equation*}
    n=\frac{n}{\gcd{(n,m)}} \text{, entonces } \gcd{(m,n)}=1
\end{equation*}
3. Sea $l$ el mínimo común múltiplo de $m$ y $n$. Sea $l/m=k$. Explicar por qué $(a^{m})^{k}=e$
\\
\\
$(a^{m})^{k}=a^{l}, l=nq$, si y solo si $a^{l}=a^{nq}=(a^{n})^{q}=e^{q}=e$
\\
\\
4. Probar que si $(a^m)^t=e$, entonces $n$ es un factor de $mt$. Entonces $mt$ es un múltiplo común de $m$ y $n$. Concluir que:
\begin{equation*}
    l \leq mt
\end{equation*}
donde $l=mcm(m,n)$
\\
\\
Se tiene por teorema que $mt=nq$, entonces $mt$ es factor común de $m$ y $n$, luego, por definición, $l$ es el mínimo entero positivo factor común entre $m$ y $n$. Entonces $l \leq mt$.
\\
\\
4. Concluir que el orden de $a^{m}$ es $\operatorname{mcm(m,n)}/m$
\\
\\
Se tiene que $(a^{m})^{k}=e$, con $k$ minimal, si y solo si $mk=nq$, entonces $\operatorname{mcm}(m,n)/m=k$, entonces, $\operatorname{ord}(a^{k})=\operatorname{mcm}(m,n)/m$
\\
\\
\textbf{H. Relación entre el orden de $a$ y el orden de cualquier raíz $k$-ésima de $a$}
\\
\\
1. Sea $a$ de orden $12$. Probar que si $a$ posee una raíz cuadrática, sease $a=b^{3}$ para algún $b \in G$, entonces $b$ posee orden 36.
\\
\\
$a^{12}=b^{36}=e$
\\
\\
2. Sea $a$ de orden $6$, si $a$ posee una raíz cuarta en $G$, $a=b^4$. ¿Cuál es el orden de $b$?
\\
\\
$a^6=b^{24}=e$
\\
\\
3. Sea $a$ de orden $10$, $a=b^6$, ¿cuál es el orden de $b$?
\\
\\
$a^{10}=b^{60}=e$
\\
\\
4. Sea $a$ de orden $n$. Si $a=b^{k}$, explicar por que el orden de $b$ es factor de $nk$.
\\
\\
$a^{n}=(b^{k})^{n}=b^{kn}=e$, por teorema, $kn=\operatorname{ord}(b)q$
\end{document}