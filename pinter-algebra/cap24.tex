\documentclass{article}
\usepackage{amsmath}
\usepackage{amssymb}
\usepackage{graphicx}
\usepackage{hyperref}
\usepackage{dsfont}
\usepackage[margin=2cm]{geometry}
\begin{document}
\title{\textbf{Anillos de Polinomios}}
\date{}
\maketitle
\subsection*{Capítulo 24}
\textbf{A. Computaciones Elementales en Dominios de Polinomios}
\\
\\
Para ahorrar notación, denotamos las clases de equivalencia $\bar{n}$ como $n$.
\\
\\
a) $a(x)+b(x)=(2x^2 +3x +1)+(x^3 +5x^2 +x)=x^3 + 7x^2 + 4x +1$
\begin{equation*}
\begin{aligned}
    &\mathbb{Z}[x]: \quad x^3 + 7x^2 + 4x +1 \\
    &\mathbb{Z}_5[x]: \quad x^3+2x^2 +4x +1 \\
    &\mathbb{Z}_6[x]: \quad x^3 +x^2 +4x +1 \\
    &\mathbb{Z}_7[x]: \quad x^3 +4x +1
\end{aligned}
\end{equation*}
b) $a(x)-b(x)=(2x^2 +3x +1)-(x^3 +5x^2 +x)=-x^3 -3x^2 +2x +1$, esto se cumple en todos los anillos dados.
\\
\\
c) $a(x)b(x)=(2x^2 +3x +1)(x^2 +5x^2 +x)=2x^5 +13x^4 +18x^3 +8x^2 +x$
\begin{equation*}
\begin{aligned}
    &\mathbb{Z}[x]: \quad 2x^5 +13x^4 +18x^3 +8x^2 +x \\
    &\mathbb{Z}_5[x]: \quad 2x^5 + 3x^4 +3x^3 +3x^2 +x \\
    &\mathbb{Z}_6[x]: \quad 2x^5 + x^4 +2x^2 +x \\
    &\mathbb{Z}_7[x]: \quad 2x^5 + 6x^4 + 4x^3 +x^2 +x
\end{aligned}
\end{equation*}
2. 
\begin{equation*}
\begin{aligned}
    &\mathbb{Z}[x]: \quad x^3 + x^2 +x +1 = (x^2 +3x +2)(x-2)+(5x+5) \\
    &\mathbb{Z}_5[x]: \quad x^3 +x^2 +x +1 = (x^2+3x +2)(x+3)
\end{aligned}
\end{equation*}
3.
\begin{equation*}
\begin{aligned}
    &\mathbb{Z}[x]: \quad x^3 +2 = (2x^2 +3x +4)(\frac{x}{2}-\frac{3}{4})+(\frac{x}{4}+5) \\
    &\text{El problema aquí es que se está dividiendo por un polinomio que no es mónico} \\
    &\text{por lo que toca multiplicar por $4$ para tener una expresión bien definida} \\
    \\
    &\mathbb{Z}_3[X]: \quad x^3 +2 = (2x^2 +3x +4)(2x) + (x+2) \\
    &\text{Se tiene que $\mathbb{Z}_3$ sí es un campo. Hay que encontrar el inverso de 2 en este} \\
    &\text{anillo para que al restar quede un polinomio mónico.} \\
    \\
    &\mathbb{Z}_5[x]: \quad x^3 +2 = (2x^2 +3x +4)(3x +3)+4x 
\end{aligned}
\end{equation*}
4. Sea $k$ par. Vamos a probar por inducción.
\\
\\
a)
\\
\\
Sea $k=2$:
\begin{equation*}
    x^2 +1 = x(x+1) +(-x +1)
\end{equation*}
Supongamos que se cumple para $k=2(n-1)$:
\begin{equation*}
    x^{2(n-1)} +1 = (x+1)q(x)
\end{equation*}
Sea $k=2n$:
\begin{equation*}
\begin{aligned}
    &x^{2n}+1 - x^{2n -1}(x+1) = -x^{2n-1} +1 \\
    & -x^{2n-1} +1 - (-x^{2n-2})(x+1) = x^{2(n-1)}+1
\end{aligned}
\end{equation*}
Por hipótesis inductiva se tiene que $x^{2(n-1)}+1$ es múltiplo de $(x+1)$, por lo que:
\begin{equation*}
    x^{2n} +1 = (x^{2n-1} + x^{2n-2})(x-1)+(x-1)q(x)= (x+1)(x^{2n-1} + x^{2n-2} + q(x))
\end{equation*}
$x+1$ es factor de $x^n +1$ para todo $n$ par.
\\
\\
b) 
\\
\\
Sea $k=2$:
\begin{equation*}
    x^2 +x +1 = x(x+1)+1
\end{equation*}
Supongamos que se cumple para todo entero menor o igual a $k=2(n-1)$:
\begin{equation*}
    x^{2(n-1)}+x^{2(n-1)-1} + \dots + 1 = (x+1)q(x)
\end{equation*}
Sea $k=2n$
\begin{equation*}
\begin{aligned}
    &x^{2n} + \dots + 1 - (x+1)(x^{2n-1}) = x^{2n-3} + \dots +1 \\
\end{aligned}
\end{equation*}
Por hipótesis inductiva, $x^{2n-3} + \dots + 1$ es múltiplo de $(x+1)$, entonces:
\begin{equation*}
    x^{2n} + \dots + 1 = (x+1)q_{0}(x)+(x+1)(x^{2n-1}) = (x+1)(q_{0}(x)+x^{2n-1})
\end{equation*}
\\
\\
5.
\\
\\
a) Es trivial para $k=1$. Supongamos q
\\
\\
6.
\begin{equation*}
    (3x^2 + 4x +m)(ax^2 +bx +c)= 3ax^4 + 3bx^3 + 4ax^3 +4bx^2 +4cx +amx^2 +bmx + cm =6x^4 +50
\end{equation*}
Igualando coeficientes:
\begin{equation*}
\begin{aligned}
    &3a = 6 \\
    &3b+4a = 0 \\
    &3c +4b +am = 0 \\
    & 5c + bm = 0 \\
    &cm = 50
\end{aligned}
\end{equation*}
Con estas ecuaciones vemos que $b=-8/3$, pero esto no puede ser. ya que estamos en $\mathbb{Z}[x]$
\\
\\
7.
\begin{equation*}
\begin{aligned}
    (x^2 +1)(x^3 + ax^2 +bx +c)&=x^5 +ax^4 +bx^3 +cx^2 +x^3 +ax^2 +bx +c \\
    &= x^5 +ax^4 + (b+1)x^3 +(a+c)x^2 +bx +c \\
    &= x^5 +5x +6
\end{aligned}
\end{equation*}
Igualando clases de congruencia:
\begin{equation*}
\begin{aligned}
    &a \equiv 0 \pmod{n} \\
    &(b+1) \equiv 0 \pmod{n} \\
    &a+c \equiv 0 \pmod{n} \\
    &b \equiv 5 \pmod{n} \\
    & c \equiv 6 \pmod{n}
\end{aligned}
\end{equation*}
De la segunda ecuación se obtiene que $b+1 \equiv 5+1 \equiv 0 \pmod{n}$. Por lo que $n=2,3,6$, viendo las demás ecuaciones, se puede apreciar que ningún valor arroja una contradicción.
\\
\\
\textbf{B. Problemas Involviendo Conceptos y Definiciones}
\\
\\
1. $x^8 +1 = x^3 +1 \Longleftrightarrow x^8 = x^3 \Longleftrightarrow x^3(x^5-1) = 0$
\\
\\
Las soluciones a esta ecuación son $x=0,1$. Entonces, los polinomios no son iguales, ya que tendrían que ser equivalentes para los valores $x=2,3,4$ también.
\\
\\
2.
\\
\\
3. Los polinomios de grado $2$ o menor en $\mathbb{Z}_5[x]$ son:
\begin{equation*}
    a_2x^2 + a_1 x + a_0 \text{ para } a_i \in \{ 0,1,2,3,4\}
\end{equation*}
De estos, hay $5^3 =125$ polinomios distintos. Los polinomios de grado $1$ o $0$ en $\mathbb{Z}_5[x]$ son:
\begin{equation*}
    b_1 x + b_0 \text{ para} b_i \in \{0,1,2,3,4 \}
\end{equation*}
De estos hay $5^2=25$ elementos distintos. Luego la cantidad de polinomios cuadráticos son: $5^3-5^2=125-25=100$.
\\
\\
Para el caso general, los polinomios de grado $m$ o menos son de la forma:
\begin{equation*}
    a_n x^n + a_{n-1} x^{n-1} + \dots + a_0
\end{equation*}
cada coeficiente es del conjunto $\{0,1,\dots,n-1 \}$. Entonces la cantidad de polinomios diferentes es $m^n$. El mismo argumento dice que hay $m^{n-1}$ polinomios distintos de grado $m-1$ o menor. Entonces, la cantidad de polinomios de grado $m$ es: $m^n - m^{n-1}$.
\\
\\
4. Sea $A$ dominio entero.
\begin{equation*}
\begin{aligned}
    &a) \quad (x+1)^2 = x^2 +2x +1 = x^2 +1 \Longrightarrow 2x = 0 \Longleftrightarrow \operatorname{char}{A}=2 \\
    &b) \quad (x+1)^4 = x^4 + 4x^3 +6x^2 +4x + 1 = x^4 +1 \Longrightarrow 2(2x^3 +3x^2 + 2x)=0 \Longleftrightarrow \operatorname{char}{A}=2 \\
    &c) \quad (x+1)^6= x^6 +6x^5 + 15x^4 + 20x^3 + 15x^2 + 6x + 1 = x^6+2x^3 +1 \Longrightarrow 3(2x^5 + 5x^4 + 6x^3 + 5x^2 + 2x) \Longleftrightarrow \operatorname{char}{A}=3
\end{aligned}
\end{equation*}
5. 
\begin{equation*}
\begin{aligned}
    &a) \quad (ax+b)(cx+d)=acx^2+(ad+bc)x + bd \\
    &\text{Entonces podemos pedir que los coeficientes cumplan que,} \\
    &ac \equiv 0 \pmod{8} \\
    &ad+bc \equiv 0 \pmod{8} \\
    &bd \equiv 0 \pmod{8}
\end{aligned}
\end{equation*}
Las ecuaciones se pueden realizar pidiendo que ningún coeficiente sea nulo, una solución sería: $a=b=1$ y $c=d=8$. Entonces $(x+1)(8x+8)=8x^4 + 16x +8 =0$. Otras soluciones se ven bajo factorización.
\begin{equation*}
\begin{aligned}
    &b) \quad (ax+1)(bx+1)=abx^2 + (a+b)x +1 \\
    &ab \equiv 0 \pmod{8} \\
    &a+b \equiv 0 \pmod{8}
\end{aligned}
\end{equation*}
Podemos pedir que $a \equiv -b \pmod{8} \Longrightarrow b^2 \equiv 0 \pmod{8} \Longrightarrow b \equiv 4 \pmod{8}$. Reemplazando, obtenemos que $a \equiv 4 \pmod{8}$. Es $(4x+1)(4x+1)=16x^2 + 8x + 1 = 1$.
\\
\\
6. Supongamos que $x$ es invertible en $A[x]$, entonces existe $p(x)$ de grado $n$ tal que $xp(x)=1$, entonces $\deg{xp(x)}=\deg{1}=\deg{x}+\deg{p(x)}=1+n=0$, pero esto significa que $n=-1$, pero por definición, el grado siempre es positivo.
\\
\\
7.
\\
\\
\textbf{D. Dominios $A[x]$ Donde $A$ Posee Característica Finita}
\\
\\
1. Para un polinomio $a(x)=\sum_{i=0}^{n}a_i x^i$, cada $a_i \in \{0,\dots,n \}$ pertenece a $A$, por lo que posee característica $p$, por lo que la característica se preserva ya que $p \cdot a_i = 0$.
\\
\\
2. Un ejemplo de un dominio entero infinito con característica finita es $Z_n$, en donde en este anillo es miembro $x^i$ para todo $i \in \mathbb{N}$.
\\
\\
3. Ver el ejercicio A5, es una generalización de este ejercicio.
\\
\\
4. Por el teorema del binomio:
\begin{equation*}
    (x+c)^p = \sum_{k=0}^{p} \binom{p}{k} x^k c^{p-k}
\end{equation*}
Se tiene que $\binom{p}{k}=\frac{p(p-1)(p-2)\dots (p-k+1)}{k!}$, es decir, cada término es múltiplo de $p$, por lo que cada término se cancela al tener un anillo característica $p$, excepto el primer y último término. Entonces $(x+c)^p=x^p + c^p$.
\\
\\
6.
\begin{equation*}
    (a(x)+b(x))^{p} = \sum_{k=0}^{p} \binom{p}{k} a^{k}(x)b^{p-k}(x)
\end{equation*}
Se tiene el mismo argumento de antes, como $A$ posee características $p$ se anula cada coeficiente menos el primer y último coeficiente. Luego:
\begin{equation*}
\begin{aligned}
    (a_0 + a_1x + \dots + a_n x^n)^{p} &= a_{0}^p + (a_1x + \dots + a_n x^n)^p \\
    &= a_{0}^{p} + a_{1}^{p}x^p + (a_{2}x^{2}+\dots+a_nx^{n})^{p} \\
\end{aligned}
\end{equation*}
Repitiendo este proceso inductivamente se tiene que:
\begin{equation*}
    (a_0 + a_1x + \dots + a_n x^n)^{p}=a_{0}^{p}+a_{1}^{p}x^{p}+\dots+a_{n}^{p}x^{np}
\end{equation*}
\\
\\
\textbf{E. Subanillos e Ideales en A[x]}
\\
\\
1. Mostrar que si $B$ es un subanillo de $A$, entonces $B[x]$ es un subanillo de $A[x]$
\\
\\
Sean $b_{1}(x)=\sum_{i=0}^{n}b_{1,i}x^{i}$ y $b_{2}(x)=\sum_{j=0}^{n}b_{2,j}x_{j}$. La suma y la multiplicación se definen como:
\begin{equation*}
\begin{aligned}
    &a. \quad b_{1}(x)+b_{2}(x)=\sum_{k=0}^{n}(b_{1,k}+b_{2,k})x^{k} \\
    &b. \quad b_{1}(x)b_{2}(x)=\sum_{k=0}^{2n} \left( \sum_{i+j=k} b_{1,i}b_{2,j}\right) x^{k}
\end{aligned}
\end{equation*}
Luego, la suma y multiplicación de elementos en $B$ es cerrada en $B$, por que la operación en el anillo de polinomios es cerrada.
\\
\\
2. Sea $a(x)=\sum_{i=0}^{n}a_{i}x^{i}$ con $a_{i} \in A$ y $b(x)=\sum_{j=0}^{m}b_{j}x^{j}$ con $b_{j} \in B$:
\begin{equation*}
    a(x)b(x)=\sum_{k=0}^{n+m} \left(\sum_{i+j=k} a_{i}b_{j} \right) x^{k}
\end{equation*}
\\
\\
Como $\sum_{i+j=k}a_{i}b_{j} \in B$ al ser $B$ ideal, entonces $a(x)b(x) \in B[x]$
\\
\\
3. Sea $S=\{ \sum_{i=0}^{n}a_{i}x_{i} \in A[x] \mid a_{i}=0 \quad \forall i \equiv 1 \pmod{2} \}$
\\
\\
Sean $p(x)=\sum_{k=0}^{n}a_{k}x^{k}$ y $q(x)=\sum_{l=0}^{n}b_{l}x^{l}$
\begin{equation*}
\begin{aligned}
    &a. \quad p(x)+q(x)=\sum_{j=0}^{n} (a_{i}+b_{i})x^{i} \\
    &b. \quad p(x)q(x)=\sum_{k=0}^{2n} \left(\sum_{i+j=k}a_{i}b_{i} \right)x_{k}
\end{aligned}
\end{equation*}
Para la suma ambos coefientes no son nulos cuando $i$ es par. Para la multiplicación la suma de los multiplicación de coeficientes no es cero cuando $i+j=k$ par, esto solo pasa cuando $i$ y $j$ son pares a la vez. La condición no se cumple para indices impares, ya que en la multiplicación la suma de impares da par, por lo que se sale del conjunto.
\\
\\
4. $J = \{ \sum_{i=0}^{n}a_{i}x^{i} \in A[x] \mid a_{0}=0\}$. Sea $a(x)=a_{1}x+a_{2}x^{2}+\dots+a_{n}x^{n} \in J$ y $b(x)=b_{0}+b_{1}x+\dots+b_{n}x^{n}$. $a(x)b(x) = a_{1}x(b_{0}+b_{1}x+\dots+b_{n}x^{n})+\dots+a_{n}x^{n}(b_{0}+b_{1}x+\dots+b_{n}x^{n})$. Por lo que $\deg a(x)b(x) \geq 1$
\\
\\
5. Sea $a(x)=a_{0}+a_{1}x+\dots+a_{n}x^{n} \in J$ y $b(x)=b_{0}+b_{1}x+\dots+b_{n}x^{n} \in A[x]$. Luego $a(x)b(x)=a_{0}(b_{0}+b_{1}x+\dots+b_{n}x^{n})+a_{1}x(b_{0}+b_{1}x+\dots+b_{n}x^{n})+\dots+a_{n}x^{n}(b_{0}+b_{1}x+\dots+b_{n}x^{n})$. Entonces $(a_{0}+\dots+a_{n})(b_{0}+\dots+b_{n})=0$. La implicancia sigue.
\\
\\
6.- $A[x]/J$ dominio entero.
\\
\\
\textbf{F. Homomorfismos de Dominios de Polinomios}
\\
\\
Sea $A$ un dominio integral.
\\
\\
1. Sean $a(x)=\sum_{i=0}^{n}a_{i}x^{i}$ y $b(x)=\sum_{j=0}^{m}b_{j}x^{j}$ con $n \leq m$, tiene que:
\begin{equation*}
\begin{aligned}
    &a. \quad h\left(a(x)+b(x)\right)=h\left(\sum_{i=0}^{n}a_{i}x^{i} + \sum_{j=0}^{m}b_{j}x^{j} \right) = h\left(\sum_{k=0}^{n} (a_{k}+b_{k}x^{k}) + \sum_{k=n+1}^{m} b_{k}x^{k} \right) = a_{0}+b_{0} = h\left(\sum_{i=0}^{n}a_{i}x^{i}\right) + h\left(\sum_{j=0}^{m}b_{j}x^{j}\right) \\
    &b. \quad h\left( \sum_{k=0}^{n+m} \left(\sum_{i+j=k}a_{i}b_{j} \right)x^{k}\right) = a_{0}b_{0}=h\left(\sum_{i=0}^{n}a_{i}x^{i} \right)h\left(\sum_{j=0}^{m}b_{j}x^{j} \right)
\end{aligned}
\end{equation*}
Kernel:
\\
\\
\begin{equation*}
\begin{aligned}
    &\ker h = \{ a(x) \in A[x] \mid h(a(x)) = 0 \} \\
    &\Longleftrightarrow h\left( \sum_{k=0}^{n} a_{k}x^{k} \right) = 0 \\
    &\Longleftrightarrow a_{0} = 0 \\
    &\Longleftrightarrow a(x) = a_{1}x + \dots + a_{n}x^{n} \\
    &\Longleftrightarrow xR(x) \quad \text{; } R(x) \in A[x]
\end{aligned}
\end{equation*}
Sobreyectividad:
\\
\\
Se tiene que para $a \in A$ se puede definir el polinomio constante $a(x)=a \in A[x]$ tal que $h(a(x))=h(a)=a$
\\
\\
2. Por punto anterior el kernel es de la forma $xq(x)$ con $q(x) \in A[x]$, es decir $xq(x) \in (x)$
\\
\\
3. Usando el primer teorema del isomorfismo se tiene que existe una biyección $\varphi$ entre el espacio cocientado por el kernel y la imagen. Se tiene que la imagen es todo $A$ (argumento de sobreyectividad). Luego $A[x]/(x) \xrightarrow{\varphi} A$
\\
\\
4. Sean $a(x)=\sum_{i=0}^{n}a_{i}x^{i}$ y $b(x)=\sum_{j=0}^{m}b_{j}x^{j}$ tal que $n \leq m$.
\\
\\
Homomorfismo:
\begin{equation*}
\begin{aligned}
    a.\text{ } g(a(x))+g(b(x))=(a_{0}+\dots+a_{n})+(b_{0}+\dots+&b_{m}) = (a_{0}+b_{0})+(a_{1}+b_{1})+\dots+(a_{n}+b_{n})+b_{n+1}+\dots+b_{m} \\
    &=g((a_{0}+b_{0})+(a_{1}+b_{1})x+\dots+(a_{n}+b_{n})x^{n}+b_{n+1}x^{n+1}+\dots+b_{m}x^{m}) \\
    &=g(a(x)+b(x)) 
\end{aligned}
\end{equation*}
\begin{equation*}
    b. \quad g(a(x))g(b(x))= \left(\sum_{i=0}^{n}a_{i}x^{i} \right) \left(\sum_{j=0}^{m}b_{j}x^{j} \right) = \left(\sum_{k=0}^{n+m} \sum_{i+j=k}a_{i}b_{j} \right)=g\left( \sum_{k=0}^{n+m}\left(\sum_{i+j=k}a_{i}b_{j}\right)x^{k}\right) = g(a(x)b(x))
\end{equation*}
Sobreyectividad: Para todo $a \in A$ se puede definir el polinomio constante $a(x)=a$ donde $g(a(x))=g(a)=a$
\\
\\
Kernel:
\begin{equation*}
\begin{aligned}
    &\ker g = \{ a(x) \in A[x] \mid g(a(x))=0 \} \\
    &\Longleftrightarrow g(a_{0}+a_{1}x+\dots+a_{n}x^{n}) = 0 \\
    &\Longleftrightarrow a_{0}+a_{1}+\dots+a_{n}=0
\end{aligned}
\end{equation*}
5.- Sean $a(x)=\sum_{i=0}^{n}a_{i}x^{i}$ y $b(x)=\sum_{j=0}^{m}b_{j}x^{j}$ tal que $n \leq m$.
\\
\\
Homomorfismo:
\begin{equation*}
\begin{aligned}
    &a. \quad h(a(x)+b(x))=\sum_{k=0}^{n}(a_{k}+b_{k})c^{k}x^{k} + \sum_{k=n+1}^{m}b_{k}c^{k}x^{k}=h\left(\sum_{i=0}^{n}a_{i}x^{i} \right) + h\left( \sum_{j=0}^{m}b_{j}x^{j}\right) = h(a(x))+h(b(x)) \\
    &b. \quad h(a(x))h(b(x)) = \left( \sum_{i=0}^{n}a_{i}c^{i}x^{i}\right) \left( \sum_{j=0}^{m}b_{j}c^{j}x^{j} \right) = \sum_{k=0}^{n+m} \left( \sum_{i+j=k}a_{i}b_{j} \right) c^{k}x^{k} = h\left( \sum_{k=0}^{n+m} \left( \sum_{i+j=k}a_{i}b_{j} \right)x^{k}\right) = h(a(x)b(x))
\end{aligned}
\end{equation*}
Kernel:
\begin{equation*}
\begin{aligned}
    &\ker h = \{ a(x) \in A[x] \mid h(a(x))=0\} \\
    &\Longleftrightarrow \sum_{i=0}^{n}a_{i}c^{i}x^{i}=0 \\
    &\Longleftrightarrow a_{i} = 0 \quad \forall i \in \{ 1, \dots, n\} \\
    &\Longleftrightarrow a(x) \equiv 0
\end{aligned}
\end{equation*}
6.
\\
\\
$\Longrightarrow$) Sea $h$ un automorfismo, es decir, un isomorfismo entre mismos anillos, en particular se tiene que es una biyección, por lo que $\ker h = (0) \Longleftrightarrow a(x) \cong 0$, suponiendo que $c$ no es invertible, entonces $c=0$m por lo que $h(a(x))=a(cx)=a(0)$, por lo que $\ker$ contendía los polinomios divisibles por $x$, contradiciendo la inyectividad.
\\
\\
Por sobreyectividad existe $a \in A[x]$ constante, se puede definir $c^{-1}a$ en donde $c(c^{-1}a) \mapsto a$. Esto solo pasa si $c$ es invertible.
\\
\\
$\Longleftarrow$) Sea $c$ invertible. Como $h^{-1}$ un homomorfismo tal que $\varphi(a(x))=a(c^{-1}x)$ (probar que es homomorfismo es directo). Hay que ver que es isomorfismo y la inversa de $h$.
\\
\\
\begin{equation*}
    \ker \varphi = \{ a(x) \in A[x] \mid \varphi(a(x))=0\} \Longleftrightarrow \sum_{i=0}^{n}a_{i}(c^{-1})^{i}x^{i}=0 \Longleftrightarrow a_{i} = 0
\end{equation*}
\begin{equation*}
    \mathrm{Im}\text{ } \varphi = \{ a(x) \in A[x] \mid \varphi(p(x))=a(x)\} \Longleftrightarrow \forall a(x) \in A[x], \exists p(x) \in A[x] \text{ con } p(x)=\sum_{i=1}^{n} a_{i}c^{i}x^{i} \Longleftrightarrow \mathrm{Im} \text{ } = A[x] 
\end{equation*}
Ver que $\varphi \equiv h^{-1}$ es directo, sea $a(x)=\sum_{i=0}^{n}a_{i}x^{i}$
\begin{equation*}
\begin{aligned}
    &\varphi\left( h(a(x))\right)=\varphi\left( \sum_{i=0}^{n}a_{i}c^{i}x^{i} \right) = \sum_{i=0}^{n}a_{i}x^{i}=a(x) \\
    &h\left( \varphi(a(x))\right)=a(x)
\end{aligned}
\end{equation*}
Entonces $\varphi \equiv h^{-1}$. Por lo que $h$ automorfismo.
\\
\\
\textbf{G. Homomorfismos de Dominios Polinomiales inducidos por un Homomorfismos de Anillos de Coeficientes}
\\
\\
1. Sean $a(x)=\sum_{i=0}^{n}a_{i}x^{i}$ y $b(x)=\sum_{j=0}^{m}b_{j}x^{j}$ tal que $n \leq m$.
\begin{equation*}
\begin{aligned}
    \bar{h}(a(x))+\bar{h}(b(x)) &= \bar{h}\left( \sum_{i=0}^{n}a_{i}x^{i} \right) + \bar{h}\left( \sum_{j=0}^{m}b_{j}x^{j} \right) \\
    &= \sum_{i=0}^{n}h(a_{i})x^{i} + \sum_{j=0}^{m}h(b_{j})x^{j} \\
    &= \sum_{i=0}^{n}h(a_{i}+b_{i})x^{i}+\sum_{i=n+1}^{m}b_{i}x^{i} \\
    &=\bar{h}(a(x)+b(x))
\end{aligned}
\end{equation*}
\begin{equation*}
    \bar{h}(a(x)b(x))=\bar{h}\left( \sum_{k=0}^{n+m} \left( \sum_{i+j=k}a_{i}b_{j} \right)x^{k} \right) = \sum_{k=0}^{n+m}h\left( \sum_{i+j=k}a_{i}b_{j}\right) x^{k} = \sum_{k=0}^{n+m} \left(\sum_{i+j=k}h(a_{i})h(b_{j}) \right) x^{k} = \bar{h}(a(x)) \bar{h}(b(x))
\end{equation*}
2. Descubrir el kernel de $\bar{h}$
\\
\\
\begin{equation*}
    \ker \bar{h} = \{ a(x) \in A[x]\ \mid \bar{h}(a(x))=0\} \Longleftrightarrow \sum_{i=0}^{n} h(a_{i})x^{i} = 0 \Longleftrightarrow h(a_{i}) = 0 \quad \forall i \in \{1,\dots,n\} \Longleftrightarrow a_i = 0 \Longleftrightarrow a(x) \equiv 0
\end{equation*}
3.
\\
\\
$\Longrightarrow$) Sea $\bar{h}$ sobreyectivo, entonces para cualquier polinomio $b(x) \in B[x]$ existe $a(x) \in A[x]$ tal que $\bar{h}(a(x))=b(x)$. Entonces para cualquier polinomio constante $b(x)=b$ existe un polinomio constante $a(x)=a$  tal que $\bar{h}(a)=b$. Por lo que $h$ sobreyectivo.
\\
\\
$\Longleftarrow$) Sea $\bar{h}$ sobreyectivo, entonces siempre puedo encontrar $a(x) \in A[x]$ para $b(x) \in B[x]$, donde $\bar{h}\left( \sum_{i=0}^{n} a_{i}x^{i}\right)=\sum_{j=0}^{n}b_{j}x^{j}$, dado que $h(a_{i})=b_{j}$ por sobreyectividad.
\\
\\
4- 
\\
\\
$\Longrightarrow$) Supongamos que $\bar{h}$ es inyectivo. Supongamos que $a \in \ker h$, entonces considerando el polinomio constante $a(x)=a$, se tiene que $\bar{h}(a(x))=h(a)=0$, como $\bar{h}$ es inyectivo, esto implica que $a(x)=0$, es decir $a = 0$, por lo que $h$ es inyectivo. ($\ker h = (0)$)
\\
\\
$\Longleftarrow)$ Supongamos que $h$ es inyectivo. Sea $a(x)=\sum_{i=0}^{n}a_{i}x^{i} \in \ker \bar{h}$. Por definición se tiene que:
\begin{equation*}
    \bar{h}\left( \sum_{i=0}^{n}a_{i}x^{i}\right) = \sum_{i=0}^{n}h(a_{i})x^{i}
\end{equation*}
Esto implica que $h(a_{i})=0$ para todo $i$, como $h$ es inyectivo, esto implica que $a_{i}=0$, es decir $a(x)=0$, por lo que $\ker \bar{h}=(0)$
\\
\\
5.-
\\
Si $a(x)$ es factor de $b(x)$, entonces $b(x)=a(x)q(x)$ con $q(x) \in A[x]$, aplicando el mapeo se tiene que:
\begin{equation*}
    \bar{h}(b(x))=\bar{h}(a(x)q(x))=\bar{h}(a(x))\bar{h}(q(x))
\end{equation*}
\\
\\
6.-
\\
Si $h: \mathbb{Z} \to \mathbb{Z}_{n}$ es el homomorfismo natural, se tiene que:
\begin{equation*}
    \ker \bar{h} = \{ a(x) \in \mathbb{Z}_n[x] \mid \bar{a(x)} = 0 \} \Longleftrightarrow \bar{h}\left( \sum_{i=0}^{n} a_{i}x^{i} \right) = \sum_{i=0}^{n} h(a_{i})x^{i}=0 \Longleftrightarrow h(a_{i})=\bar{0} \Longleftrightarrow \bar{a_{i}} \equiv 0 \text{ } (n) \Longleftrightarrow n \mid a_{i}
\end{equation*}
(Aparte, $\ker \bar{h}=n \mathbb{Z}$)
\\
\\
7.
\\
Sea \( \bar{h}: \mathbb{Z} \to \mathbb{Z}_n[x] \), se tiene que \( \ker \bar{h} = n \mathbb{Z}[x] \). Como \( a(x)b(x) \in n \mathbb{Z}[x] \), se cumple que \( n \mid a(x)b(x) \). Entonces, \( n \mid a(x) \) o \( n \mid b(x) \), ya que $n$ es primo.
\\
\\
\textbf{H. Polinomios en Varias Variables}
\\
\\
1.- Vamos a demostrar por inducción.
\\
\\
Sea $k=1$. Entonces hay que mostrar que $A[x_{1}]$ es dominio integral, sean $a(x)$ y $b(x)$ polinomios no nulos, debemos mostrar que $a(x)b(x)$ no es nulo. Sea $a_{n}$ el coeficiente principal de $a(x)$ y $b_{m}$ el coeficiente principal de $b(x)$. Por definición $a_{n} \neq 0 \neq b_{m}$. Entonces $a_{n}b_{n} \neq 0$ por que $A$ es dominio entero. Sigue que $a(x)b(x)$ posee almenos un coeficiente no nulo. Entonces no es el polinomio nulo
\\
\\
Supogamos que se cumple para $k=n-1$ ($A[x_{1},\dots,x_{i-2}]$ dominio entero, entonces $A[x_{1},\dots,x_{o-1}]$ es dominio entero)
\\
\\
Sea $k=n$. Sean $a(x_{1},\dots,x_{n})$ y $b(x_{1},\dots,x_{n})$ polinomios en $n$ variables no nulos. Por hipótesis se tiene que $A[x_{1},\dots,x_{n-1}]$ es dominio integral. Sean $a(x_{1},\dots,x_{n})=\sum_{i=0}^{m_{1}}p_{i}(x_{1},\dots,x_{n})x_{n}^{i}$ y $b(x_{1},\dots,x_{n})=\sum_{j_0}^{m_{2}}q_{j}(x_{1},\dots,x_{n-1})x_{n}^{j}$. Hay que mostrar que $a(x)b(x) \neq 0$.
\begin{equation*}
    a(x)b(x) = \sum_{k=0}^{m_{1}+m_{2}} \left( \sum_{i+j=k} p_{i}q_{j}(x_{1},\dots,x_{n-1}) \right) x_{n}^{k}
\end{equation*}
Si $p_{i}(x_{1},\dots,x_{n-1}) \neq 0 \neq q_{j}(x_{1},\dots,x_{n-1})$, entonces $a(x)b(x)$ no es nulo 
ya que hay unos $i,j$ tal que 
\\
$p_{i}(x_{1},\dots,x_{n-1})q_{j}(x_{1},\dots,x_{n-1}) \neq 0$.
\\
\\
2. 
\\
a) Sea $p(x,y)=\sum_{(i,j)\in \mathbb{N}^{2}}a_{i,j}x^{i}y^{j}$ con $a_{i,j} \in A$. El grado de $p(x,y)$ se define como:
\begin{equation*}
    \deg(p(x, y)) = \max \{ i + j \mid a_{i,j} \neq 0 \}
\end{equation*}
b) Se tiene que todos los polinomios de grado $\leq 3$ en $\mathbb{Z}_{3}[x,y]$ son combinaciones lineales de $\{1,x,y,x^{2},xy,y^{2},x^3,x^2y,xy^{2},y^{3} \}$
\\
\\
3. Sea $p(x,y)=\sum_{(i,j) \in \mathbb{N}^{2}}a_{i,j}x^{i}y^{j}$ y $q(x,y)=\sum_{(i,j) \in \mathbb{N}^{2}}b_{(i,j)x^{i}y^{j}}$. La suma se puede definir como:
\begin{equation*}
    p(x,y)+q(x,y)=\sum_{(i,j)\in \mathbb{N}^{2}}(a_{i,j}+b_{i,j})x^{i}y^{j}
\end{equation*}
\\
\\
\[
\text{Sean } p(x, y) = \sum_{(i, j) \in \mathbb{N}^2} a_{i,j} x^i y^j 
\quad \text{y} \quad 
q(x, y) = \sum_{(k, \ell) \in \mathbb{N}^2} b_{k,\ell} x^k y^\ell,
\]
donde \( a_{i,j}, b_{k,\ell} \neq 0 \) solo para un número finito de pares \((i, j)\) y \((k, \ell)\).
\\
La multiplicación de \( p(x, y) \) y \( q(x, y) \) está definida como:
\[
p(x, y) \cdot q(x, y) = \sum_{(m, n) \in \mathbb{N}^2} \left( \sum_{\substack{(i, j), (k, \ell) \in \mathbb{N}^2 \\ i + k = m, \, j + \ell = n}} a_{i,j} b_{k,\ell} \right) x^m y^n.
\]
\\
\textbf{I. 
Cuerpos de Cocientes de Polinomios}
\\
\\
3. 
\\
a) Sea $a(x) \in \ker \bar{h} = \{p(x) \in A(x) \mid \bar{h}(p(x))=0 \}$, sea tal que:
\begin{equation*}
    \bar{h} \left(\sum_{k=0}^{n}a_{k}x^{k} \right) = \sum_{k=0}^{n} h(a_{k})x^{k} = 0 \Longleftrightarrow h(a_{k})=0
\end{equation*}
Como $h$ es isomorfismo, entonces $h(a_{k})$ si y solo si $a_{k}=0 \quad \forall k \in \{i,\dots,n\}$, es decir $a(x) \equiv 0$. $\ker \bar{h}=(0)$
\\
\\
b) Sea $b(x)=\sum_{k=0}^{n}b_{k}x^{k}$, como $h$ es un isomorfismo, existe $a_{k}$ tal que $h(a_{k})=b_{k}$ para todo $k$. Equivalente a decir que existe siempre $a(x)$ tal que $b(x)=\bar{h}(a_{k})$.
\\
\\
\textbf{J. Algoritmo de División: Unicidad del Cuociente y del Resto}
\\
\\
Suponer que $a(x)=b(x)q_{1}(x)+r_{1}(x)=b(x)q_{2}(x)+r_{2}(x)$. Entonces $b(x)(q_{1}(x)-q_{2}(x))+(r_{1}(x)-r_{2}(x))=0$m esta expresión la reduciré a $b(x)q(x)+r(x)=0$, donde $\deg b(x) > \deg r(x)$, esto es equivalente a que $\deg b(x) > \deg r_{1}(x)\text{ , } \deg r_{2}(x)$. A su ves se tiene que $b(x)q(x)$ es múltiplo de $b(x)$. Suponiendo que $q(x) \neq 0$, se tiene que $\deg b(x)q(x) \geq \deg b(x)$. Esto quiere decir que $\deg r(x) > \deg b(x)$. Esto es una contradicción.
\end{document}

