\documentclass{article}
\usepackage{amsmath}
\usepackage{amssymb}
\usepackage{graphicx}
\usepackage{hyperref}
\usepackage{mathrsfs}
\usepackage{dsfont}
\usepackage{hyperref}
\usepackage[margin=2cm]{geometry}

\newcommand{\R}{\mathbb{R}}
\renewcommand{\P}{\mathbb{P}}
\newcommand{\A}{\mathbb{A}}
\newcommand{\C}{\mathbb{C}}
\newcommand{\Z}{\mathbb{Z}}

\newcommand{\br}[1]{\left( #1 \right)}

\begin{document}
\subsection*{C. Consecuencias Elementales del Teorema de Langrange}
Sea $G$ un grupo finito. Probar:
\\
\\
1. Si $G$ posee orden $n$, entonces $x^{n}=e$ para todo $x \in G$
\\
\\
Sea $x$ de orden $m$, entonces $\langle x \rangle$ posee $m$ elementos, por lo que $m \mid n$, es decir $n=km$, luego $x^{n}=x^{km}=(x^{m})^{k}=e^{k}=e$.
\\
\\
2.  Sea $G$ de orden $pq$ con $p,q$ primos. Entonces $G$ es ciclico o cada elemento $x \neq e$ posee orden $p$ o $q$.
\\
\\
Si $|G|=pq$, los subgrupos proios poseen $p$ o $q$ elementos, es decir para $H \subseteq G$, $|H|=p,q$, es decir, es cíclico, por lo que (sin pérdida de generalidad, tomando $p$) $h \in H \Longrightarrow h^{p}=e$. Si $|H|=pq$, entonces existe $a$ tal que $a^{pq}=e$ y el orden de $a$ es $pq$, por lo que $|\langle a \rangle|=pq$, entonces $\langle a \rangle=G$.
\\
\\
3. Sea $G$ de orden $4$. Entonces $G$ es cíclico o cada elemento de $G$ es su propio inverso. Concluir que cada grupo de orden $4$ es abeliano.
\\
\\
Sea $|G|=4$, entonces $|H|=2$ por Lagrange, es decir es cíclico donde $h^{2}=e \Longleftrightarrow h=h^{-1}$, luego $G=\{e,a,b,c\}$, pero $ab \neq a,b$ por cancelación, por lo que $c=ab$, luego por las mismas razones $c=ba$, es decir $ab=ba$. Luego si $|H|=4$, $G=H$ y $G=\{e,a,a^{2},a^{3} \}$ y $\varphi: \Z_{4} \to G$ comprende un isomorfismo. Por lo que es abeliano.
\\
\\
4. Si $G$ posee un elemento de orden $p$ y un elemento de orden $q$, donde $(p,q)=1$, entonces el orden de $G$ es múltiplo de $pq$.
\\
\\
Se tiene que $p\mid |G|$ y $q \mid |G|$, luego $p,q$ son coprimos, por lo que ninguno posee factores en común del otro, por lo que $pq \mid |G|$ es decir $|G|=pqk$.
\\
\\
5. Si $G$ posee un elemento de orden $k$ y un elemento de orden $m$, entonces $|G|$ es un múltiplo de $\operatorname{lcm}(k,m)$.
\\
\\
Si $(k,m) \geq 1$, $k \mid |G|$ y $m \mid |G|$, por lo que $|G|$ es múltiplo común de $k$ y $m$, es decir $\operatorname{lcm}(k,m) \mid |G|$.
\\
\\
6. Sea $p$ primo. En cualquier grupo finito, el número de elementos de orden $p$ es múltiplo de $p-1$
\subsection*{D. Más Consecuencias del Teorema de Lagrange}
Sea $G$ un grupo finito, sea $H$ y $K$ un subgrupo de $G$. Probar:
\\
\\
1. Suponer $H \subseteq K$. Entonces $(G:H)=(G:K)(K:H)$.
\\
\\
Por definición $(G:H)=\frac{|G|}{|H|}$, luego como $H \subseteq K$, $(K:H)=\frac{|K|}{|H|} \Longrightarrow |H|=\frac{|K|}{(K:H)}$.
\\
\\
Juntando todo: $(G:H)=\frac{|G|}{|K|} \cdot (K:H)=(G:K)(K:H)$.
\\
\\
2. El orden de $H \cap K$ es divisor común del orden de $H$ y el orden de $K$.
\\
\\
Hay que probr que $H \cap K$ es subgrupo de $H$ y $K$.
\\
\\
Sea $x \in H\cap K$, luego $x^{-1} \in H$ y $x^{-1} \in K$, por lo que $x^{-1} \in H\cap K$.
\\
Sea $a,b^{-1} \in H \cap K$, luego $a,b^{-1} \in H$ y $a,b^{-1} \in K$, por lo que $ab^{-1} \in H,K$, luego $ab^{-1} \in H \cap K$.
\\
\\
Por Lagrange, $|H \cap K| \mid H,K$
\\
\\
3. Sea $H$ de orden $m$ y $K$ de orden $n$, donde $m,n$ son coprimos. Entonces $H \cap K=\{e \}$. 
\\
\\
Como $m,n$ son coprimos, por lo que el "único factor cómún" es el $1$, por el punto anterior esto quiere decir que $|H \cap K|=1$, luego $H \cap K = \{e\}$.
\\
\\
4. Suponer que $H$ y $K$ no son equivalentes y ambos poseen como orden el mismo número primo $p$. Entonces $H \cap K=\{e\}$
\\
\\
Como $|H \cap K| \mid p$ y $|H \cap K|$ es divisor propio del orden de los grupos, $|H \cap K|=1$, luego $H \cap K =\{e\}$.
\\
\\
5. Suponer que $H$ posee índice $p$ y $K$ posee índice $q$ donde $p$ y $q$ son primos distintos, entonces el índice de $H \cap K$ es múltiplo de $pq$.
\\
\\
Como $(G:H)=p$ y $(G:K)=q$, por ley de torres (de grupos) $(G:H\cap K)=(G:K)(K: H \cap K)$ y $(G:H \cap K)=(G:H)(H: H \cap K)$, por lo que $p,q \mid (G: H \cap K)$, es decir $pq \mid (G: H \cap K)$.
\\
\\
6. Si $G$ es grupo abeliano de orden $n$ y $m$ es un entero tal que $m$ y $n$ son coprimos, entonces la función $f(x)=x^{m}$ es un automorfismo de $G$.
\\
\\
\textit{Homomorfismo:} $f(xy)=(xy)^{m}=x^{m}y^{m}=f(x)f(y)$.
\\
\textit{Inyectividad:} Sea $x^{m}=y^{m}$, luego $(xy^{-1})^{m}=e$, por
ejercicio anterior, $m \mid n$, pero son coprimos, por lo que la única opcion es que $xy^{-1}=e$, es decir $x=y$.
\\
\textit{Sobreyectividad:} Directamente para todo $x^{m}$ se puede definir $x$.
\subsection*{E. Propiedades Elementales de Cosets}
Sea $G$ un grupo y $H$ un subgrupo de $G$. Sean $a,b$ elementos de $G$. Probar:
\\
\\
1. $Ha=Hb$ si y solo si $ab^{-1} \in H$.
\\
\\
Si $Ha=Hb$, entonces $h_{1}a=h_{2}b$, luego $h_{1}ab^{-1}=h_{2}$, es decir $Hab^{-1}=H$, $ab^{-1}\in H$. Las implicancias se leen en sus respectivas direcciones.
\\
\\
2. $Ha=H$ si y solo si $a \in H$.
\\
\\
Es la misma demostración anterior tomando $b=e$
\\
\\
3. Si $aH=Ha$ y $bH=Hb$, entonces $(ab)H=H(ab)$.
\\
\\
$abh_1=a(h_2b)=(h3a)b$, luego $(ab)H=H(ab)$
\\
\\
4. Si $aH=Ha$, entonces $a^{-1}H=Ha^{-1}$.
\\
\\
$a^{-1}h_1=(h_1^{-1}a)^{-1}=(ah_{2}^{-1})^{-1}=h_{2}a^{-1} \Longleftrightarrow a^{-1}H=Ha^{-1}$
\\
\\
5. Si $(ab)H=(ac)H$, entonces $bH=cH$
\\
\\
$abh_{1}=ach_{2} \Longleftrightarrow bh_{1}=ch_{2} \Longleftrightarrow bH=cH$.
\\
\\
6. El número de cosets derechos de $H$ es equivalente al número de cosets izquierdos de $H$. 
\\
\\
Esto viene directo de las biyecciones $\varphi: H \to aH$ y $\phi: H \to Ha$ y de la definición de índice.
\\
\\
7. Si $J$ es un subgrupo de $G$ tal que $J=H \cap K$, entonces para cualquier $a \in G$, $Ja=Ha \cap Ka$. Concluir que si $H$ y $K$ poseen indice finito en $G$, entonces su intersección también posee índice finito.
\\
\\
Esto ya fué probado, la primera implicacia es trivial por doble contención, Luego la segunda implicancia es directa ya que el orden es divisor común del orden de $H$ y $K$.
\\
\\
\subsection*{F. Estudio de Todos los Grupos de Seis Elementos}
\textit{Sea $G$ cualquier grupo de orden $6$. Por Teorema de Cauchy $G$ posee un elemento de orden $2$ y un elemento $b$ de orden $3$. Por ejercicio anterior, los elementos:}
\begin{equation*}
    \{e,a,b,b^{2},ab,ab^2{} \}
\end{equation*}
\textit{son distintos, y como $G$ posee solo seis elementos, estos son todos los elementos de $G$. Entonces $ba$ es uno de los elementos, $e,a,b,b^{2},ab$ o $ab^{2}$.}
\\
\\
1. Probar que $ba$ no puede ser igual a $e,a,b$ o $b^{2}$, entonces $ba=ab$ o $ba=ab^{2}$.
\\
\\
(i) $ba=e \Longleftrightarrow ba^{2}=a \Longleftrightarrow b=a$. No
\\
(ii) $ba=a \Longleftrightarrow ba^{2}=a^{2} \Longleftrightarrow b=e$. No
\\
(iii) $ba=b \Longleftrightarrow a=e$. No.
\\
(iv) $ba=b^{2} \Longleftarrow b=a$.
\\
\\
Luego las únicas posibilidades son $ba=ab$ y $ba=ab^{2}$.
\\
\\
2. Si $ba=ab$. Probar que $G \cong \Z_{6}$
\\
\\
Si $ba=ab$, se tiene que $a,b$ conmutan, entonces $\langle ab \rangle = \langle a \rangle \times \langle b \rangle=\Z_{2} \times \Z_{3} \cong \Z_{6}$
\\
\\
3. Si $ba=ab^{2}$. Probar que $G \cong S_{3}$
\\
\\
Como $a^{2}=e$, $b^{3}=e$ y $ba=ab^{2} \Longleftrightarrow a^{-1}ba^{2}=b^{2}=b^{-1}$, los elementos describen $S_{3}$. Luego $G \cong S_{3}$
\subsection*{G. Estudio de Todos los Grupos de 10 Elementos}
Sea $G$ un grupo cualquiera de orden $10$.
\\
\\
1. Razonar como en el ejercicio anterior para mostrar que $G=\{e,a,b,b^{2},b^{3},b^{4},ab,ab^{2},ab^{3},ab^{4} \}$ donde $a$ posee orden $2$ y $b$ posee orden $5$.
\\
\\
Sea $|G|=10$, por teorema de Cauchy, $G$ posee un elemento $a$ de orden $2$ y un elemento $b$ de orden $5$, luego todos los elementos de la forma $a^{i}b^{j}$, $0 \leq i \leq 2$, $0 \leq j \leq 5$ son diferentes al ser $2,5$ coprimos.
\\
\\
2. Probar que $ba$ no puede ser igual a $e,a,b^{2},b^{3}$ o $b^4$.
\\
\\
(i) $ba=e \Longleftrightarrow b=a^{-1}=a$. No
\\
(ii) $ba=a \Longleftrightarrow b=e$. No
\\
(iii) $ba=b \Longleftrightarrow a=e$. No
\\
(iv) $ba=b^{2} \Longleftrightarrow a=b$. No
\\
(v) $ba=b^4 \Longleftrightarrow a=b^{3}$. No
\\
\\
3. Probar que si $ba=ab$, entonces $G \cong \Z_{10}$
\\
\\
$G$ es generado por $ab$, como conmutan y $2$ y $5$ son coprimos, se tiene que es equivalente al producto directo:
\begin{equation*}
    \langle ab \rangle \cong \langle a \rangle \times \langle b \rangle \cong \Z_2 \times \Z_5 \cong \Z_{10}
\end{equation*}
\\
\\
4. Si $ba=ab^{2}$. Probar que $ba^{2}=a^{2}b^{4}$ y concluir que $b=b^4$, lo que es imposible, pues $b$ posee orden $5$. Entonces $ba \neq ab^{2}$.
\\
\\
Si $ba=ab^{2}$, entonces $ba^{2}=baa=ab^{2}a=abba=abab^{2}=a(ba)b^{2}=aab^{2}b^{2}=a^{2}b^{4}$. Luego $ba^{2}=a^{2}b^{4}$, $b=b^{4}$. Pero esto es imposible, pues $b$ tiene orden $5$. La suposición inicial es falsa.
\\
\\
5. Si $ba=ab^{3}$. Probar que $ba^{2}=a^{2}b^{3}=a^{2}b^{4}$. Concluir que $b=b^4$ y llegar a una contradicción.
\\
\\
$ba=ab^{3} \Longrightarrow ba^{2}=ab^{3}a=ab^{2}(ba)=ab^{2}ab^{3}=ab(ba)b^{3}=abab^{3}b^{3}=a(ba)b^{6}=aab^{3}b^{6}=a^{2}b^{9}$. Luego $ba^{2}=a^{2}b^{9}=a^{2}b^{4}$, por lo que $b=b^{4}$. Contradicción, la suposición inicial es falsa.
\\
\\
Probar que si $ba=ab^{4}$, entonces $G \cong D_{5}$ ($D_5$ siendo el grupo de simetrías del pentágono).
\\
\\
Si $a^{2}=e$ y $b^{5}=e$ y $ba=ab^{4} \Longrightarrow a^{-1}ba=b^{-1}$. Luego $G \equiv D_5$
\subsection{H. Estudio de Todos los Grupos de Ocho Elementos}
Sea $G$ un grupo cualquiera de orden $8$. Si $G$ posee un elemento de orden $8$, entonces $G \equiv Z_{8}$. Entonces asumir que $G$ no posee elementos de orden $8$, entonces todos los elementos $\neq e$ en $G$ poseen orden $2$ o $4$.
\\
\\
1. Si cada $x \neq e$ en $G$ posee orden $2$, sea $a,b,c$ los tres elementos. Probar que $G=\e,a,b,c,ab,bc,ac,abc \}$. Concluir que $G \cong \Z_2 \times \Z_2 \times \Z_2$.
\\
\\
Todo elemento de la forma $a^{i}b^{j}c^{k}$ con $0 \leq i,j,k \leq 2$. Luego como todo elemento es de orden $2$, en particular permutan entre sí ya que $a=a^{-1}$. Luego $G \cong \Z_2 \times \Z_2 \times \Z_2$
\\
\\
2. Asumir que en $Hb$ hay un elemento de orden $2$ (sea $b$) este elemento). Si $ba=a^{2}b$, probar que $b^{2}a=a^{4}b^{2}$, entonces $a=a^{4}$, lo cual es imposible. Concluir que $ba=ab$ o $ba=a^3b$.
\\
\\
$b^2a=b(ba)=ba^2b=(ba)ab=a^{2}(ba)b=a^4b^2$, luego $b^2a=a^4b^2 \Longleftrightarrow a=a^4$, luego $a=e$, pero esto es una contradicción ya que $|G|=8$.
\\
\\
3. Sea $b$ como en (2). Probar que si $ba=ab$, entonces $G \cong \Z_4 \times \Z_2$.
\\
\\
Si $|a|=4$ y $|b|=2$ y todo elemento conmuta, luego $G \cong \langle a \rangle \times \langle b \rangle \cong \Z_4 \times \Z_2$
\\
\\
4. Sea $b$ como en la parte (2). Probar que si $ba=a^3b$, entonces $G \cong D_4$.
\\
\\
Si $|a|=4$ y $|b|=2$, luego si $ba=a^3b \Longleftrightarrow bab^-1=a^-1$, por lo que $G \cong D_4$
\subsection*{I. Elementos Conjugados}
Si $a \in G$, un conjugado de $a$ es cualquier elemento de la forma $xax^{-1}$, donde $x \in G$. Probar:
\\
\\
1. La relación "$a$ es eqvuialente a el conjugado de $b$" es una relación de equivalencia en $G$ (Escribir $a \sim b$ para denotar la relación)
\\
\\
(i) $a \sim a$: $a=xax^{-1} \Longleftrightarrow x^{-1}ax=a$
\\
(ii) Si $a \sim b$: $a=xbx^{-1} \Longleftrightarrow x^{-1}ax=b$, entonces $b \sim a$
\\
(iii) Si $a \sim b$ y $b \sim c$, $a=xbx^{-1}$ y $b=xcx^{-1}$, luego $a=x(xcx^{-1})x^{-1}=x^{2}c(x^{2})^{-1}$.
\\
\\
Luego $\sim$ es una relación de equivalencia.
\\
\\
\textit{(*) Para cualquier elemento $a \in G$, el centralizador de $a$, denotado por $C_{a}$, es el conjunto de los elementos en $G$ que conmutan con $a$, eso es:}
\begin{equation*}
    C_{a}=\{x \in G: xa=ax \} = \{x \in G: xax^{-1}=a \}
\end{equation*}
2. Para cualquier $a \in G$, $C_{a}$ es subgrupo de $G$.
\\
\\
(i) $x,y \in C_{a}$, entonces $a=xax^{-1}$, $a=yay^{-1}$, entonces $a=x(yay^{-1})x^{-1}=(xy)a(xy)^{-1}$, por lo que $xy \in C_{a}$
\\
(ii) Sea $x \in C_{a}$, entonces $a=xax^{-1} \Longleftrightarrow a=x^{-1}ax$, por lo que $x^{-1} \in C_a$
\\
(iii) $a=ea=eae=eae^{-1}$, luego $e \in C_a$
\\
\\
3. $x{-1}ax=y^{-1}ay$ si y solo si $xy^{-1}$ conmuta con $a$ si y solo si $xy^{-1} \in C_a$
\\
\\
Si $x^{-1}ax=y^{-1}ay \Longleftrightarrow a=xy^{-1}ayx^{-1}=xy^{-1}a(x^{-1})^{-1}$, entonces $xy^{-1}$ comuta con $a$ y $xy^{-1} \in C_a$.
\\
\\
4. $x^{-1}ax=y^{-1}ay$ si y solo si $C_ax=C_ay$.
\\
\\
Como $x^{-1}ax=y^{-1}ay$, entonces $xy^{-1} \in C_a$, esto es equivalente a que $C_axy^{-1}=C_a$, luego $C_ax=C_ay$.
\\
\\
5. Hay una correspondencia uno-a-uno entre el conjunto de todos los conjugados de $a$ y el conjunto de todos los cosets de $C_a$.
\\
\\
Sea $\varphi: \{xax^{-1} \mid a \in G \} \to \{yC_a \mid y \in G\}: x^{-1}ax \mapsto xC_a$
\\
\\
\textit{Inyectividad}: Si $xC_a=yC_a$, entonces $x^{-1}ax=y^{-1}ay$ por punto anterior
\\
\textit{Sobreyectividad}: Para $xC_a \in \{yC_a \mid y \in G \}$ siempre se puede definir $x^{-1}ax$ tal que $\varphi(x^{-1}ax)=xC_{a}$
\\
\\
Luego se tiene una correspondencia uno-a-uno (biyectividad).
\\
\\
6. El número de conjugados distintos de $a$ es igual a $(G:C_{a})$, el índice de $C_{a}$ en $G$. Entonces el tamaño de cada clase es factor de $G$.
\\
\\
Como $C_a$ subgrupo. Por teorema y por la biyección:
\\
\\
$\frac{|G|}{|C_{a}|}=|\{yC_{a} \mid y \in G \}|=(G:C_{a})$
\subsection*{J. Grupo Actuando en un Conjunto}
Sea $A$ un conjunto y sea $G$ cualquier subgrupo de $S_A$. $G$ es un grupo de permutaciones de $A$, decimos que es un grupo actuando en el conjunto $A$. Asumir que $G$ es un grupo finito, si $u \in A$, la órbita de $u$ (con respecto a $G$) es el conjunto:
\begin{equation*}
    O(u)= \{g(u) \mid g \in G \}
\end{equation*}
1. Se define una relación $\sim$ en $A$ por $u \sim v$ si y solo si $g(u)=v$. Probar que $\sim$ es una relación de equivaqlencia en $A$ y que las órbitas son sus clases de equivalencia.
\\
\\
(i) $u \sim u \Longleftrightarrow g(u)=u=g^{-1}(u)$.
\\
(ii) Si $x \sim y$, entonces $g(x)=y$, luego $g \in G$, por lo que $g^{-1} \in G$, luego $g^{-1}(g(x))=x=g^{-1}(y)$, por lo que $y \sim x$.
\\
(iii) Si $x \sim y$ y $y \sim z$: $g_{1}(x)=y$ y $g_{2}(y)=z$, luego $g_{2}(g_{1}(x))=g_{2}(y)=z$. Por lo que $x \sim z$
\\
\\
\textit{Si $u \in A$, el estabilizador de $u$} es el conjunto $G_{u}=\{g \in G: g(u)=u \}$, este es el conjunto de todas las permutaciones en $G$ que dejan $u$ fijo.
\\
\\
2. Probar que $G_{u}$ es un subgrupo de $G$.
\\
\\
(i) $e \in G_{u}$, ya que $e \in G$ y $e(u)=u$
\\
(ii) Sea $g_1,g_2 \in G_u$: $g_{1}(g_{2}(u))=g_1(u)=u$
\\
(iii) Sea $g \in G_{u}:$ $g(u)=u \Longleftrightarrow g^{-1}(u)=u$ 
\\
\\
3. AAAAAAAAAAAAAAAAA
\\
\\
4. Sea $f,g \in G$. Probar que $f$ y $g$ están en el mismo coset izquierdo de $G_u$ si y solo si $f(u)=g(u)$.
\\
\\
$f,g \in xG_{u} \Longleftrightarrow fG_{u}=gG_{u}$, esto implica que $fg^{-1} \in G_{u}$, es decir $fg^{-1}(u)=u$, luego $f(u)=g(u)$.
\\
\\
5. Usar la parte $4$ para mostrar que el número de elementos en $O(u)$ es igual a el índice de $G_{u}$ en $G$:
\\
\\
Sea $\varphi: \{gG_{u} \mid g \in G \} \to O(u): fG_{u} \mapsto f(u)$
\\
\\
\textit{Inyectividad}: $f(u)=g(u) \Longleftrightarrow fG_{u}=gG_{u}$ (punto anterior)
\\
\textit{Sobreyectividad}: Para $f(u)=v$ en $O(u)$, siempre se puede definir $fG_{u}$.
\\
\\
Luego, hay una bitección entre $\{gG_{u} \mid g \in G \}$ y $O(u)$, luego:
\begin{equation*}
    \frac{|G|}{|G_{u}|}=(G:G_{u})=|\{gG_{u} \mid g \in G \}| = |O(u)|
\end{equation*}
\end{document}